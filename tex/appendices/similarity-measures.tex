%!TEX root = ../main.tex;
\chapter{Les mesures de similarité}
\label{annexe:similarity-measurement}

L'objectif des mesures de similarité sémantique est d'évaluer la
proximité sémantique entre les concepts (auxquels les termes des
requêtes et documents sont rattachés). Et comme le matching des
services web revient à trouver les similarités entre les concepts
décrivant les propriétés de services web, nous avons jugé utile
d'explorer les différentes méthodes de recherche de similarité
utilisées dans le domaine de la recherche d’information, en
particulier.

  % Soit $\Phi$ est l'ensemble de toutes les services référencés dans un
  % annuaire de services donné. Pour une requête \verb|R|.

  % \begin{mydef}[Alignement]
  %   on parle souvent de l'alignement des ontologies. C'est un ensemble
  %   de correspondances entre deux ou plusieurs
  %   ontologies. L'alignement est la sortie d’un processus de matching
  % \end{mydef}

  % \begin{mydef}[Mapping]
  %   chercher des correspondances pour établir des transformations
  %   entre deux objets de même nature mais pas de même forme. Par
  %   conséquent, le mapping utilise les résultats du matching pour
  %   effectuer les transformations des objets.
  % \end{mydef}

  % Comme on le voit, tous ces termes prennent bien en compte la notion
  % de recherche de correspondance entre des concepts, dans certains
  % contextes ils sont utilisés indifféremment.  Pour les services web,
  % on parlera du matchnig ou matchmaking pour exprimer la mise en
  % correspondance entre deux entités, et du mapping pour exprimer la
  % transformation ou conversion des types de données.

  % Nous distinguons deux techniques de matching : matching syntaxique
  % et matching sémantique . Ils diffèrent dans la façon dont sont
  % calculés les éléments correspondants et sur le type de relation de
  % similitude \verb|M| utilisée:

  %   \subsubsection{Matching syntaxique}
  %   \label{sec:matching-syntactique}
  %   Utilisé pour l'extraction automatique de motifs textuels écrits en
  %   langage naturel en utilisant la méthode d'étiquetage, différentes
  %   techniques sont utilisées :
  %   \renewcommand{\descriptionlabel}[1]{\hspace{0.1cm}\textbullet~\textsf{#1}}
  %   \begin{description}
  %   \item [Techniques basées string (à mots)]: s méthodes prennent
  %     l’avantage de la structure des mots qui est considérée comme
  %     étant une séquence de lettres. Elles devraient, par exemple,
  %     reconnaitre une similarité entre Book et textBook. Certaines
  %     fonctions de calcul de distance convertissent une paire de
  %     chaine de caractères en un nombre réel. Si le nombre est petit
  %     (par rapport à un seuil) cela indique une grande similarité
  %     entre ces deux chaines de caractères (ex : distance de Hamming).

  %   \item [Techniques basées langage]: Ces méthodes considère les noms
  %     ou bien un terme comme une séquence de mots dans un langage
  %     naturel (peer-review, reviewed-paper).  Ces méthodes utilisent
  %     les techniques de traitement du langage naturel pour reconnaitre
  %     les similarités entre les mots d’un terme par exemple : les
  %     techniques de tokenization (segmentation), lemmatisation...etc.
  %     Techniques utilisant des ressources linguistiques: comme les
  %     thesaurus sont utilisées pour faire correspondre des mots en se
  %     basant sur les relations linguistiques entre eux (synonymes,
  %     hyponyme). Par exemple dans le cas du matching entre les
  %     services web on fait correspondre les noms d’opérations et les
  %     noms des paramètres qui sont considérés comme des mots de
  %     langage naturel.
  %   \end{description}

  %   \subsubsection{Matching sémantique}
  %   \label{sec:matching-semanique}
  %   Consiste à chercher des correspondances sémantiques à partir des
  %   concepts, et non pas des étiquettes. Il ne suffit pas d'examiner
  %   la syntaxe des mots, mais leurs significations ou leurs
  %   interprétations dans le domaine d’application. Les ontologies de
  %   domaine et les techniques basées modèle sont utilisées dans ce
  %   type de matching.

  %   \begin{description}
  %   \item [Ontologie de domaine spécifique]: spécifique peut être
  %     utilisée comme source externe. Elles se focalisent sur un
  %     domaine particulier et utilisent des termes qui sont propres à
  %     ce domaine et qui ne sont pas liés aux concepts similaires dans
  %     d’autres domaines. Par exemple dans le domaine de transport, on
  %     trouve : bus, avion, taxi...etc. On peut trouver des termes qui
  %     sont syntaxiquement identiques mais possédant des
  %     interprétations différentes dans des domaines différents.

  %   \item [Techniques basées modèle]: Appelées aussi les méthodes de
  %     déduction, les algorithmes manipulent les entrées basées sur
  %     leurs interprétations sémantiques. Si deux entités sont
  %     similaires alors ils partagent les mêmes interprétations. Parmi
  %     ces méthodes on trouve les techniques de satisfiabilité
  %     propositionnelle [Giu, 2003A] Dans le cas du matching des
  %     services web, ces derniers sont décrits à l’aide de supports
  %   \end{description}

  %   Dans le cas du matching de services web, ces derniers sont décrits
  %   à l'aide de supports de description sémantique (OWL-S par
  %   exemple). Ces descriptions contiennent des informations clés pour
  %   leur mise en correspondance (les paramètres fonctionnels et non
  %   fonctionnels).

\begin{mydef}[Similarité entre concepts]
  Une similarité $\sigma$: C $\times$ C $\rightarrow$ [0,1] est une
  fonction de couples de concepts vers un nombre compris entre 0 et 1
  exprimant le degré de similarité entre deux concepts, tel que:
  \SpecialItem
  \begin{itemize}
  \item x $\in$ C, $\sigma$(x,x) = 1
  \item x, y $\in$ C, $\sigma$(x,y) = $\sigma$(y,x)
  \end{itemize}
\end{mydef}

Les ontologies définissent les propriétés des concepts et les
relations entre eux. Une de ces relations est utilisée dans notre
application est \textsc{is–a} (est-un). Cette relation est utilisée
pour extraire des taxonomies à partir des ontologies

\begin{mydef}[Similarité basée sur la taxonomie]
  Soit C l'ensemble de concepts définis dans une ontologie. On dit que
  T est une taxonomie et on écrit T(C, $\leq$) tel que c $\leq$ c'
  signifie que c \textit{is–a} c' (le concept c est subsumé par le
  concept c').
\end{mydef}

De nombreuses approches ont été proposées pour évaluer la similarité
sémantique entre deux concepts. Ces approches se divisent en trois
catégories [Sli, 2006]: les approches basées sur les arcs, les
approches basées sur le contenu informationnel et les approches
hybrides.

\section{Mesures de similarité syntaxique}
\label{sec:syntactic-sim}

  \subsection{Approches basées sur les chaînes de caractères}
  \label{sec:syntactic-sim:string}

  La similarité de Jaccard

  Distance de Jaro

  Distance Jaro Winkler

  Distance de Bigram

  \subsection{Les approches basées sur les tokens}
  \label{sec:syntactic-sim:tokens}

  \subsection{Approches hybrides}
  \label{sec:syntactic-sim:hybrides}

\section{Mesures de similarité sémantique}
\label{sec:semantic-sim}

De nombreuses approches ont été proposées pour évaluer la similarité
sémantique entre deux concepts. Ces approches se divisent en trois
catégories [Sli, 2006]: les approches basées sur les arcs, les
approches basées sur le contenu informationnel et les approches
hybrides.

  \subsection{Approches basées sur les arcs}
  \label{sec:semantic-sim:arcs}

  Le calcul des distances dans l’ontologie est basé sur un graphe de
  spécialisation des objets.  Dans chaque graphe, la distance de
  l’ontologie doit être caractérisée par le plus court chemin qui fait
  intervenir un ancêtre commun ou le plus petit généralisant,
  connectant potentiellement deux objets à travers des descendants
  communs. Parmi les travaux classifiés sous cette bannière on trouve:\\

  \renewcommand{\descriptionlabel}[1]{\hspace{1cm}\textbullet~\textsf{#1}}
  \begin{itemize}
  \item [la Mesure de Wu \& Palmer] Basée sur le principe suivant :
    Etant donnée une ontologie $\Pi$ formée par un ensemble de nœuds
    (Figure 28). R est la racine, X et Y les deux éléments de
    l’ontologie dont nous allons calculer la similarité. Le principe
    de calcul de similarité est basé sur les distances (N1 et N2) qui
    séparent les nœuds X et Y du nœud racine et la distance qui sépare
    le concept subsumant (CS) de X et de Y du nœud R.

    La mesure de Wu et Palmer est définie par la CS formule suivante:

    \begin{equation}
      \label{wu_palmer}
      Sim(x,y) = 2 *  \frac{N}{N_{1} + N_{2} + N_{3}}
    \end{equation}\\

    La mesure de [WuP, 1994] est intéressante mais présente une
    limite, cependant, avec cette mesure on peut obtenir une
    similarité plus élevée entre un concept et son voisinage par
    rapport à ce même concept et un concept fils, ce qui est inadéquat
    dans le cadre de la recherche de l’information.\\

  \item [La mesure de Zargayouna]
    Cette mesure de similarité est inspirée de celle de Wu-Palmer
    favorisant les concepts se trouvant dans la même hiérarchie par
    rapport au voisinage. Le lien père-fils est ainsi privilégié par
    rapport aux autres liens de voisinage en adaptant la mesure de
    Wu-Palmer. L'adaptation de la mesure est faite à travers la
    fonction de calcul du degré de spécialisation d'un concept appelé
    spectre (spec) qui mesure sa distance par rapport à l'anti-racine.

    \begin{equation}
      \label{zargayouna}
      Sim(x,y)= 2 * \frac{N}{N_{1} + N_{2} + 2 * N + 2 * Spec(x,y)}
    \end{equation}\\

    Où $Spec(x,y)=N_{1}*N_{2} *N_{3}$, tel que $N_{3}$ correspond au
    nombre maximum d'arcs qui séparent le plus petit ancêtre commun du
    concept \textit{virtuel} représentant l'anti-racine, $N_{1}$ et
    $N_{2}$ séparent les nœuds $x$ et $y$ du nœud racine.\\
  \end{itemize}

  La mesure de Resnik

  \subsection{Approches basées sur le contenu informationnel}
  \label{sec:semantic-sim:ci}

  La mesure de Resnik

  La mesure de Lin

  \subsection{Approches hybrides}
  \label{sec:semantic-sim:hybrides}

  Mesure de Jiang et Conrath

%%% Local Variables:
%%% mode: latex
%%% TeX-master: "../main"
%%% End: