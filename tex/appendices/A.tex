% !TEX root = ../main.tex;
\chapter{Le Web sémantique}
\label{annexe:semantic-web}

L'expression \emph{Web sémantique}, due à Tim Berners-Lee
\cite{berners2001semantic} au sein du \acrshort{w3c}, fait d'abord
référence à la vision du Web de demain comme un vaste espace d'échange
de ressources et de collaboration entre les êtres humains et les
machines permettant une exploitation, qualitativement supérieure, de
grands volumes d'informations et de services variés.\medskip

Le Web actuel est essentiellement \emph{syntaxique}, dans le sens que
la structure des documents (ou ressources au sens large) est bien
définie mais que son contenu reste quasi inaccessible aux traitements
machines. Il s'agit d'un Web de documents, les resources sont
structurées en \acrshort{html}, identifiés de manière unique par des
\acrshort{uri} reliés entre eux par des liens
hypertextes. L'information est essentiellement textuelle et la
strucure riche qui est généralement conçus dans des schémas
relationnelles de bases de données est presque complètement perdu dans
le processus de publication de telles données sous formes des
documents \acrshort{html}. La nouvelle génération de Web nommée
\emph{``Le Web sémantique''} a pour ambition de lever cette
difficulté. Les ressources du Web seront plus aisément accessibles
aussi bien par l'homme que par la machine, grâce à la représentation
\emph{sémantique} de leurs contenus.

\newpage
\section{La vision du Web sémantique}
\label{sec:semantic-web-vision}

\section{La description des ressources Web: RDF}
\label{sec:semantic-web-rdf}
%% Dans cette section on va parler de RDF/RDFS

\section{Langage de définition des ontologies: OWL}
\label{sec:semantic-web-owl}
% What is semantic web?

% Le Web tel que nous le connaissons aujourd'hui est encore conforme à
% la vision initiale.

% Le Web a été conçu principalement pour une utilisation par les
% humains. Néanmoins, il existe un effort visant à automatiser son
% utilisation et d'être plus accessible pour les machines.

% \cite{bartalos2011effective} The Web was primarily designed for use
% by humans. Nevertheless, there is an effort to automate its use and
% bring the Web more accessible for machines. This has brought forward
% the need for machine processable representations of semantically
% rich information. This has brought forward the need for machine
% processable representations of semantically rich information: a
% vision at the heart of the Semantic Web

% Dans un premier temps, on va essayer de clarifier la notion d'un
% service Web sémantique, puis étudier les langages émergeants qui
% permettent de décrire ce type de service Web.

% L'objectif premier du Web sémantique est de définir et lier les
% ressources du Web afin de simplifier leur utilisation, leur
% découverte, leur intégration et leur réutilisation dans le plus
% grand nombre d'applications \cite{berners2001semantic}. Le Web
% sémantique doit fournir l'accès à ces ressources par l'intermédiaire
% de descriptions sémantiques exploitables et compréhensibles par des
% machines. En effet, Les technologies du Web sémantique complètent le
% Web actuel avec des outils sémantiques. Il ne s'agit donc pas de
% créer un nouveau Web ou un Web séparé de l'existant : ce Web de
% données repose entièrement sur les technologies et concepts qui ont
% fait le succès du Web tel que nous le connaissons aujourd'hui
% \cite{bertails2010web}.

% La réalisation du Web sémantique trouve ces racines dans le
% développement des langages de balises inspiré par des travaux issus
% de la communauté AI \cite{mcilraith2001semantic}, tels que
% \textsc{OIL} \cite{fensel2001oil}, \textsc{DAML+OIL}
% \cite{horrocks2002daml+oil} et \textsc{DAML+OTN}
% \cite{mcguinness2003daml} (ces deux derniers langages font partie de
% la famille \acrshort{daml}).

% TODO refactor Ces langages ont une sémantique bien définies et
% permettent le balisage et la manipulation des taxonomies complexe et
% des relations logiques entre les entités sur le
% Web. \cite{fensel2000creating}

% \input{figs/3w_to_sws.tex}

% Cette description repose sur des ontologies. Selon Gruber
% \cite{gruber1993translation}, une ontologie est une spécification
% explicite d'une conceptualisation. Une conceptualisation est un
% modèle abstrait qui représente la manière dont les personnes
% conçoivent les choses réelles dans le monde et une spécification
% explicite signifie que les concepts et les relations d'un modèle
% abstrait reçoivent des noms et des définitions explicites. Le Web
% sémantique est devenu un domaine à part entière, preuve en est la
% création en 2001 du groupe de travail sur ce sujet par le
% \textsc{W3C}.

%%% Local Variables:
%%% mode: latex
%%% TeX-master: "../main"
%%% End:
