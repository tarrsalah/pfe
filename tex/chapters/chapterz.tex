\chapter{Vers les bases de données graphe}
\label{ch:graph-db}
Avec le développement rapide et continu de l'Internet et le cloud
computing, divers types d'applications ont émergés, ce qui augmente
l'mportance accrue de la technologie de base de données, notamment
dans les aspects suivants \cite{han2011survey}:

\begin{itemize}
\item Haute concurrente de lecture et d'écriture avec une faible
  latence.
\item Stockage et la gestion d'une grande masse de données \emph{(Big
    Data)}.
\item Haute scalabilité (évolutivité) et disponibilité.
\end{itemize}

Bien que les bases de données relationnelles ont occupé une grande
position dans le domaine de stockage de données, le modèle relationnel
commençait à montrer ses limites en faisant face au-dessus exigences
\cite{han2011survey}:

\begin{itemize}
\item Lente lecture et l'écriture.
\item Capacité limitée.
\item Difficulté d'expansion et scalabilité.
\end{itemize}

Afin de faire face aux limitations ci-dessus, une variété de nouveaux
types de bases de données ont été apparus qui s'affranchissent du
modèle relationnel pour miser sur le partitionnement horizontal et le
relâchement des contraintes d'intégrité, ce modèle émergent est appelé
le modèle \acrshort{nosql}.

Dans ce chapitre, nous exposons les différentes approches et
plateformes du stockage \textsc{NoSQL} les plus connues dans
l'industrie tout en mettant l'accent sur les bases de données graphes
(et \emph{Neo4j} particulièrement). Nous commençons d'abord par
présenter les quatre catégories majeurs des systèmes
\textsc{NoSQL}. Ensuite, nous focalisons sur les \acrshort{SGBD}
orientés graphes, leurs différents modèles et implémentations ainsi
que les langages de requêtes les plus adoptés pour les interroger.

\section{L'environnement NoSQL}
\label{sec:nosql}
% TODO: make a real definition
\acrshort{nosql} désigne une catégorie de systèmes de gestion de base
de données (\acrshort{SGBD}) qui n'est plus fondée sur l'architecture
classique des bases relationnelles. L'unité logique n'y est plus la
table, et les données ne sont en général pas manipulées avec
\textsc{SQL}. À l'origine, servant à manipuler des bases de données
géantes pour des sites web de très grande audience tels que Google
Amazon, Facebook et eBay. D'aprés Martin fowler \textit{et al.}
\cite{sadalage2012nosql}, Les caractéristiques communes des bases de
données \acrshort{nosql} sont:

\begin{itemize}
\item Ne sont pas fondées sur de le modèle relationnel classique.
\item Optimisées pour les environments distribués.
\item Généralement sous des licences \emph{Open Source}.
\item Construites pour les applications Web modernes avec une grande
  masse de données.
\item \emph{Schemaless} (il n'existe plus d'intégrité référentielle ou
  schéma prédéfini).
\end{itemize}

  \subsection{Catégories des bases de données NoSQL}
  \label{sec:cat-nosql}
  Il existe dans la mouvance \acrshort{nosql} une diversité importante
  d'approches. que nous classons en quatre grandes catégories : dépots
  clés/valeurs, bases orientées documents, orientées colonnes, et
  bases orientées graphes.

  \begin{itemize}
  \item [Dépots clés/valeurs]: Le principe des bases
    \textit{clé/valeur} est de stocker les données sous une forme
    simple: une \emph{clé } (chaîne de caractères) associé à une
    \emph{valeur} d'une forme libre (chaîne de caractère, nombre ou
    bien un objet sérialisé), cette clé est utilisée pout toutes les
    opérations à effectuer sur les données telles que l'insertion, la
    mise à jour et la supression. Bien que la structure est plus
    simple, la vitesse d'interrogation est extrêmement supérieur à la
    base de données relationnelles favorisant la scalabilité
    (\emph{scalability}) plus que la cohérence, on les retrouve très
    souvent comme système de stockage de cache ou de sessions
    distribuées, notamment là où l'intégrité relationnelle des données
    est non significative. Les solutions les plus connues, telle que
    \emph{Redis}, \emph{Riak} et \emph{Voldemort} sont principalement
    influencés par le project \emph{Dynamo} d'Amazon
    \cite{decandia2007dynamo}.

  \item [Orientées documents]: Cette famille de base de données est
    une évolution de la base de données \textit{clé/valeur} destinée
    aux applications qui gèrent des documents (généralement du format
    \textsc{JSON} ou \textsc{XML}) où chaque clé n'est plus associée à
    une valeur sous forme de bloc binaire mais à un document dont la
    structure reste libre (\textit{scheme-less}). L'avantage est de
    pouvoir récupérer, via une seule clé, un ensemble d’informations
    structurées de manière hiérarchique. ansi que le stockage de
    volumes très importants de données pour lesquelles la modélisation
    relationnelle aurait entraînée une limitation des possibilités de
    partitionnement et de réplication. les deux implémentations les
    plus populaires dans cette catégorie sont \emph{CouchDB} d'Apache
    et \emph{MongoDB}.

  \item [Orientées colonnes]: Une base de données orientée colonnes
    est une base de données qui stocke les données par colonne et non
    par ligne. L'orientation colonne permet d'ajouter des colonnes
    plus facilement aux tables (les lignes n'ont pas besoin d'être
    redimensionnées). Elle permet de plus une compression par colonne,
    efficace lorsque les données de la colonne se ressemblent. Comme
    solutions, on retrouve principalement \emph{HBase},
    \emph{HyperTable} (implémentations Open Source du modèle
    \emph{BigTable} \cite{chang2008bigtable} publié par Google) ainsi
    que \emph{Cassandra} (projet Apache qui respecte l'architecture
    distribuée de \emph{Dynamo} \cite{decandia2007dynamo} d'Amazon et
    le modèle BigTable de Google).

  \item [Orientées graphes]: Ce paradigme est le moins connu de ceux
    de la mouvance \acrshort{nosql}. Ce modèle s'appuie principalement
    sur deux concepts: d'une part l'utilisation d'un moteur de
    stockage pour les objets (qui se présentent sous la forme d'une
    base documentaire, chaque entité de cette base étant nommée
    \emph{nœud}). D'autre part, à ce modèle, vient s'ajouter un
    mécanisme permettant de décrire les relations entre les objets
    (\emph{arcs}). Principalement, ces bases de données sont nettement
    plus efficaces que leur pendant relationnel pour traiter les
    problématiques liées aux réseaux.  En effet, lorsqu'on utilise le
    modèle relationnel, cela nécessite un grand nombre d'opérations
    complexes (souvent, des jointures trop lentes) pour obtenir des
    résultats. Dans cette catégorie on peut citer \emph{Neo4j},
    \emph{OrientDB}, \emph{Titan} et \emph{AllegroGraph} comme les
    implémentations les plus répondus de ce modèle.
  \end{itemize}

    \subsection{Le théorème de CAP }
  \label{sec:cap}

  \acrshort{cap} \cite{brewer2000towards} est l'acronyme de
  \textit{``\textbf{C}onsistency, \textbf{A}vailability and
    \textbf{P}artition Tolerance''}, ou \textit{``Cohérence,
    Disponibilité et Résistance au partitionnement''}. Ce théorème
  explique qu'il est impossible qu'un système distribué satisfasse
  simultanément aux trois contraintes suivantes:

  \begin{itemize}
  \item [Cohérence]: tous les nœuds du système voient exactement les
    mêmes données au même moment.

  \item [Disponibilité]: la perte de nœuds n'empêche pas les survivants
    de continuer à fonctionner correctement.

  \item [Résistance au partitionnement]: aucune panne moins importante
    qu'une coupure totale du réseau ne doit empêcher le système de
    répondre correctement (ou encore : en cas de partitionnement ,
    chacune des partitions doit pouvoir fonctionner de manière
    autonome).
  \end{itemize}

    \subsection{Le théorème de CAP }
  \label{sec:cap}

  \acrshort{cap} \cite{brewer2000towards} est l'acronyme de
  \textit{``\textbf{C}onsistency, \textbf{A}vailability and
    \textbf{P}artition Tolerance''}, ou \textit{``Cohérence,
    Disponibilité et Résistance au partitionnement''}. Ce théorème
  explique qu'il est impossible qu'un système distribué satisfasse
  simultanément aux trois contraintes suivantes:

  \begin{itemize}
  \item [Cohérence]: tous les nœuds du système voient exactement les
    mêmes données au même moment.

  \item [Disponibilité]: la perte de nœuds n'empêche pas les survivants
    de continuer à fonctionner correctement.

  \item [Résistance au partitionnement]: aucune panne moins importante
    qu'une coupure totale du réseau ne doit empêcher le système de
    répondre correctement (ou encore : en cas de partitionnement ,
    chacune des partitions doit pouvoir fonctionner de manière
    autonome).
  \end{itemize}

    \subsection{Le théorème de CAP }
  \label{sec:cap}

  \acrshort{cap} \cite{brewer2000towards} est l'acronyme de
  \textit{``\textbf{C}onsistency, \textbf{A}vailability and
    \textbf{P}artition Tolerance''}, ou \textit{``Cohérence,
    Disponibilité et Résistance au partitionnement''}. Ce théorème
  explique qu'il est impossible qu'un système distribué satisfasse
  simultanément aux trois contraintes suivantes:

  \begin{itemize}
  \item [Cohérence]: tous les nœuds du système voient exactement les
    mêmes données au même moment.

  \item [Disponibilité]: la perte de nœuds n'empêche pas les survivants
    de continuer à fonctionner correctement.

  \item [Résistance au partitionnement]: aucune panne moins importante
    qu'une coupure totale du réseau ne doit empêcher le système de
    répondre correctement (ou encore : en cas de partitionnement ,
    chacune des partitions doit pouvoir fonctionner de manière
    autonome).
  \end{itemize}

  \input{figs/cap.tex}

  Le théorème de \acrshort{cap} stipule qu'il est impossible d'obtenir
  ces trois propriétés en même temps dans un système distribué et
  qu'il faut donc en choisir deux parmi les trois, Les bases de
  données relationnelles implémentent les propriétés de cohérence et
  de disponibilité (système \emph{CA}). Les bases de données
  \emph{NoSQL} sont généralement des systèmes \emph{CP} (cohérent et
  résistant au partitionnement) ou \emph{AP} (disponible et résistant
  au partitionnement), la figure \ref{fig:cap} présente le
  positionnement de quelque systèmes \emph{NoSQL} par rapport au
  théorème \acrshort{cap}.

%%% Local Variables:
%%% mode: latex
%%% TeX-master: "../main"
%%% End:


  Le théorème de \acrshort{cap} stipule qu'il est impossible d'obtenir
  ces trois propriétés en même temps dans un système distribué et
  qu'il faut donc en choisir deux parmi les trois, Les bases de
  données relationnelles implémentent les propriétés de cohérence et
  de disponibilité (système \emph{CA}). Les bases de données
  \emph{NoSQL} sont généralement des systèmes \emph{CP} (cohérent et
  résistant au partitionnement) ou \emph{AP} (disponible et résistant
  au partitionnement), la figure \ref{fig:cap} présente le
  positionnement de quelque systèmes \emph{NoSQL} par rapport au
  théorème \acrshort{cap}.

%%% Local Variables:
%%% mode: latex
%%% TeX-master: "../main"
%%% End:


  Le théorème de \acrshort{cap} stipule qu'il est impossible d'obtenir
  ces trois propriétés en même temps dans un système distribué et
  qu'il faut donc en choisir deux parmi les trois, Les bases de
  données relationnelles implémentent les propriétés de cohérence et
  de disponibilité (système \emph{CA}). Les bases de données
  \emph{NoSQL} sont généralement des systèmes \emph{CP} (cohérent et
  résistant au partitionnement) ou \emph{AP} (disponible et résistant
  au partitionnement), la figure \ref{fig:cap} présente le
  positionnement de quelque systèmes \emph{NoSQL} par rapport au
  théorème \acrshort{cap}.

%%% Local Variables:
%%% mode: latex
%%% TeX-master: "../main"
%%% End:

\section{Bases de données orientées graphes}
\label{sec:graph-database-overview}
% TODO: rewrite the intro
\begin{text}
  Un grand nombre de problèmes pratiques dans différentes disciplines
  peuvent être intuitivement représentés sous forme de graphes : des
  nœuds reliés par des arcs (étiqueté ou non).

  Depuis plusieurs décennies, les développeurs ont essayé de stoker
  des ensembles de données connectés, semi-structurées à l'intérieur
  des bases de données relationnelles. Mais alors que les bases de
  données relationnelles ont été initialement conçues pour codifier
  des structures tabulaires. Cependant, les données fortement
  connectées sont traitées de manière très pauvre par les bases de
  données relationnelles. Chaque opération sur une relation dans un
  graphes ou réseau résulte en une opération de jointure dans le
  \acrshort{SGBDR}, implémentée comme une opération ensembliste entre
  l'ensemble des clés primaires de deux tables - une opération lente
  et sans capacité à monter en charge alors que le nombre de t-uples
  de ces tables augmente \cite{robinson2013graph}.

  Les bases de données orientées graphes sont donc conçues pour
  modéliser des réseaux de données fortement connectées et y naviguer
  facilement en bénéficiant de performances extrêmement élevées – un
  atout qui explique leur succès auprès de
  Facebook\footnote{\url{http://www.facebook.com}},
  LinkedIn\footnote{\url{http://www.linkedin.com}} et autres réseaux
  sociaux. Ces derniers sont devenus ces dernières années l'un de cas
  d'utilisation les plus visibles des bases de données graphes.
  LinkedIn parvient ainsi facilement à afficher le degré de séparation
  entre chaque contact, qui n'est finalement que la distance entre les
  nœuds dans le graphes représentant les personnes et leurs relations.
\end{text}

% TODO: relational vs graph database
  \subsection{Techniques de persistance des bases de données graphes}
  \label{sec:persistence-tech}
  % TODO: rewrite
  Généralement, nous pouvons distinguer trois techniques majeures du
  stockage adoptés par les systèmes de gestion des bases de données
  graphes: les bases de données graphes qui utilisent des
  \acrshort{SGBDR} comme un \emph{backend}, celles qui utilisent des
  systèmes \acrshort{nosql}, et enfin, nous avons les bases de données
  graphes natives qui fournissent ses propres implémentations du
  stockage en termes des nœuds et des arcs directement.

    \subsubsection{Bases de données graphes au-dessus de d'un stockage  SQL}
    \label{sec:graphdb-over-sql}
    % TODO: rewrite this paragraph
    Une base de donnée graphe peut être stockée dans une base de
    données relationnelle. Les étiquettes et les attributs de nœuds et
    arcs peuvent être gérés séparément dans d'autres tables et renvoyé
    par des clés étrangères. l'utilisation des \acrshort{SGBDR} comme
    un moteur de stockage a quelques avantages : des systèmes
    d'indexation évolués, un support des transactions sophistiqué, et
    le langage de requêtes \emph{SQL} qui est un standard bien établi
    avec a cycle d'apprentissage rapide.

    \begin{figure}[h]
    \centering
    \includegraphics[width=0.8\textwidth]{figs/periscope-gq.eps}
    \caption{L'architecture du système Periscope/GQ \cite{tian2008periscope}}
    \label{fig:periscope-gq}
\end{figure}

%%% Local Variables:
%%% mode: latex
%%% TeX-master: "../main"
%%% End:


    Cette classe comprend \emph{GRIPP} \cite{trissl2007fast}, et
    Periscope/GQ \cite{tian2008periscope} développé par l'université
    de Michigan qui implémente un système de gestion des bases de
    données graphes comme une application au-dessus d'un moteur de
    stockage relationnel \emph{PostgreSQL} (ce qui est démontré dans
    la figure \ref{fig:periscope-gq}).

    \subsubsection{Bases de données graphes au-dessus d'un stackages NoSQL}
    \label{sec:graphdb-over-nosql}
    Plusieurs systèmes de base de données graphes emploient des
    systèmes \acrshort{nosql} comme un moteur de stockage interne
    offrant une meilleure scalabilité et un support fiable pour le
    partitionnement des données.

    \emph{HypergraphDB} \cite{hypergraphdb, iordanov2010hypergraphdb}
    est une base de donnée qui implémente le modèle de données
    \emph{``hypergraph''} où la notion d'un arc est étendue pour
    pouvoir connecter plus de deux nœuds, ce qui est particulièrement
    utile pour la modélisation des données tels que la représentation
    des connaissances, l'intelligence artificielle et la
    bio-informatique. \emph{HypergraphDB} utilise le modèle
    \textit{clé/valeur} de \emph{BerkeleyDB} \cite{berkeleydb} pour
    stocker toutes les informations relatives du graphe sous forme de
    pairs \textit{clé/valeur}, chaque objet du graphe (nœud ou arc)
    est identifié par un clé unique (appelé atome). Chaque atome est
    lié à un ensemble des atomes par une relation de type \emph{0:N}
    (zéro ou plusieurs atomes), ces relations forment également la
    structure typologique ``hypergraph''.

    \emph{OrientDB} \cite{orientdb} est un système hybride de gestion
    de base de données graphes qui combine les fonctionnalités d'une
    base orientée documents et une base orientée graphes avec une
    capacité de stockage des données structurées ou semi-structurées
    (\emph{schema-less}). il supporte aussi la répartition de charge à
    travers plusieurs machines et la réplication multi-maîtres tout en
    assurant les propriétés \acrshort{acid} de données. Pour les
    requêtes simples, \emph{OrientDB} s'appuie sur \emph{SQL} et
    utilise des langages de parcours des graphes comme \emph{Gremlin}
    afin d'éviter les jointures \emph{SQL} coûteuses pour les requêtes
    complexes.

    \emph{Titan} \cite{titan} est une base de données orientée graphes
    évolutive (\emph{scalable}) et transactionnelle, optimisée pour le
    stockage et l'interrogation des données graphes contenant des
    centaines de milliards de sommets et d'arcs à travers un
    \emph{cluster} multi-machine avec des schémas complexes du
    parcours et requêtage et une éxecution en temps réal. \emph{Titan}
    utilise une multitude des systèmes \emph{NoSQL} comme un moteur de
    stockage (\emph{backend}), par exemple, \emph{Hbase},
    \emph{Cassandra}, \emph{BerkeleyDB}, cette diversité offre une
    flexibilité en terme des caractéristiques \emph{CAP} \ref{sec:cap}
    assurées par le système.

    \subsubsection{Les bases de données graphes natives}
    \label{sec:graphdb-native}
    Les bases de données graphes natives possèdent leurs propres
    systèmes de fichiers pour stocker les données au lieu de compter
    sur des moteurs de stockage tiers. Ces bases de données sont
    optimisés pour stocker et gérer les données \textbf{fortement}
    connectées où la performance et la disponibilité sont
    primordiales.

    % TODO: translate + enhance
    \emph{Neo4j} \cite{neo4j} is a disk-based transactional graph
    database that implemented in Java. It is one the most popular
    graph databases. Besides the advantages in traversing the graph
    data like other graph databases, it highlights its full ACID
    transaction support and convenient REST server interfaces. Those
    features ensure it to be suitable for its enterprise solutions. It
    also pro- vides its own query language called Cypher, which can
    handle different kinds of queries with syntax. At the same time,
    their developers make APIs for almost all of program- ming
    languages to access their graph database.

      %TODO: enhance Allegrograph paragraph
    \emph{AllegroGraph} \cite{allegrograph} est un système performant
    de persistence des données graphes avec un stockage natif,
    implémenté initialement comme une base de \acrshort{rdf}, avec un
    support d'interrogation \acrshort{sparql} et un service
    \acrshort{rest} \cite{fielding2000architectural}. Similaire à
    \emph{Neo4j}, \emph{Allegrograph} garantis la satisfaction des
    propriétés \acrshort{acid} d'atomicité, cohérence, isolation et
    durabilité. \emph{AllegroGraph} supporte \acrshort{sparql}, RDFS++
    et Prolog pour des applications clientes multiples.

    \emph{Sparsity} \cite{sparksee} (auparavant connu sous le nom
    \emph{DEX}) est un système natif de stockage de données graphes
    persistants et temporaires. \emph{Sparsity} focalise sur la
    gestion et l'interrogation des géantes bases de données graphes à
    haute performance. En encodant les matrices d'adjacences en
    \emph{bitmap}, le système a une gestion d'espace disque compact et
    efficace afin de d'assurer une trés bonne performance.

    \emph{InfiniteGraph} \cite{infinitegraph} est un système distribué
    de gestion des bases de données graphes qui supporte des grandes
    masses évolutives de données (\emph{large-scale}) avec une
    capacité d'analyse efficace des graphes et une bonne prise en
    charge des algoritmes distribués du parcours. \emph{InfiniteGraph}
    fournis une variété d'implémentations pour des différents systèmes
    (serveurs d'application, les plateformes de \emph{cloud computing}
    et les systèmes embarqués).

  \subsection{Détails d'implémentation}
  \label{sec:graph-internals}
      \begin{figure}[!hr]
    \centering
    \begin{subfigure}[h]{0.7\textwidth}
        \centering
        \includegraphics[width=\textwidth]{figs/global-graph-index-table.eps}
        \label{fig:global-graph-index-table}
    \end{subfigure}

    \begin{subfigure}[h]{0.4\textwidth}
        \centering
        \includegraphics[width=\textwidth]{figs/global-graph-index.eps}
        \label{fig:global-graph-index}
    \end{subfigure}

    \caption{Une indexation global des données fortement connectées}
    \label{fig:gpi}
\end{figure}

%%% Local Variables:
%%% mode: latex
%%% TeX-master: "../main"
%%% End:


  \subsection{Comparaison}
  \label{sec:graphdb-comp}

 % \begin{table}[htb!]
  \centering
  \begin{tabular}{|lcccc|}
  \end{tabular}
  \newline
  \caption{Comparaison des bases de données graphes}
  \label{tab:graphdb-comp}
\end{table}

%%% Local Variables:
%%% mode: latex
%%% TeX-master: "../main"
%%% End:

\newpage\
\section{Langages des requêtes}
\label{sec:query-languages}

Les langages de requêtes ont toujours été la clé du succès des
\acrshort{SGBD}. La prévalence des \emph{\acrshort{SGBDR}} dans les
dernières décennies est étroitement couplée avec le succès du
\emph{SQL}. Des divers langages ont été définis pour exprimer des
requêtes vis-à-vis plusieurs dépôts de données, par exemple,
\emph{XQuery} \cite{boag2002xquery} et \emph{XPath}
\cite{clark1999xml} pour les bases de données \emph{\acrshort{xml}},
\emph{QQL} \cite{alashqur1989oql} pour les bases de données orientées
objets et \acrshort{sparql} \cite{prud2008sparql} pour les
triplestores (les bases de données \emph{RDF}).

\begin{figure}[!h]
  \centering
  \includegraphics[width=0.7\textwidth]{figs/query-graph-example.eps}
  \caption{Un example d'un graphe orienté}
  \label{fig:query-graph-example}
\end{figure}

%%% Local Variables:
%%% mode: latex
%%% TeX-master: "../main"
%%% End:


Cette section présente trois langages de requêtes largement supportés
par les systèmes de gestion des bases de données graphes. Le graphe
présenté par la figure \ref{fig:query-graph-example} sera utilisé pour
illustrer le syntaxe et la sémantiques de ces trois languges. Le
graphe représente le résultat d'un processus de \emph{matching}
sémantique entre cinq service Web qui forme un plan de composition.


  % definition
  % example
  % syntax
  % graph-db support

  \subsection{SPARQL}
  \label{sec:sparql}

  \acrshort{sparql} \cite{prud2008sparql} est un langage populaire de
  requêtes pour les données \acrshort{rdf}, il est reconnu comme l'une
  des technologies clés du Web sémantique. Le standard
  \acrshort{sparql} est largement utilisé pour exprimer des
  interrogations à travers diverses sources de données graphes vues
  comme des triplestores \acrshort{rdf}. Il est capable de rechercher
  des motifs de graphe (\emph{graph patterns}) ainsi que leurs
  conjonctions et leurs disjonctions. Les résultats des interrogations
  \textsc{SPARQL} peuvent être des ensembles de résultats ou des
  graphes \acrshort{rdf} qui peuvent être retournés via
  \acrshort{http} dans une variété de formats tels que \acrshort{xml},
  HTML ou \acrshort{json}

  La plupart des bases de données graphes ont un support de
  \textsc{SPARQL} soit nativement telle que \emph{Allegrograph} soit
  via des extensions comme dans le cas de \emph{Neo4j}.

  \subsection{Gremlin}
  \label{sec:gremlin}
  \emph{Gremlin} \cite{gremlin-wiki} est un langage de domaine
  spécifique (\acrshort{DSL}) de bas niveau pour le parcours des
  graphes attribués, il trouve ses applications dans les domaines de
  la recherche, l'analyse et la manipulation des bases de données
  orientées graphes qui implémentent le modèle \emph{Blueprints}
  \cite{blueprints} de données. \emph{Gremlin} \cite{gremlin-wiki} est
  un projet open source développé et maintenu par \emph{TinkerPop}.

  Le syntaxe \emph{Gremlin} est basée sur \emph{XPath} de manière à
  être capable d'exprimer des descriptions de parcours même profonds
  avec des expressions simples et compactes.

  La distribution \emph{Gremlin} (maintenu par \emph{TinkerPop}) est
  supporté par la plupart des bases de données graphes via des
  langages \emph{JVM} comme \emph{Java}, \emph{Groovy} et
  \emph{Scala}, parmi ces \acrshort{SGBD} graphes nous trouvons
  \emph{Neo4j}, \emph{Titan} et \emph{OrientDB}.

  \subsection{Cypher}
  \label{sec:cypher}
  \emph{Cypher} \cite{cypher-docs} est un langage des requêtes
  déclaratif pour interagir avec les bases des données graphes
  \emph{Neo4j}, développé et maintenu par \emph{Neo Technology}. Il
  permet d'effectuer des requêtes et mises jour du graphe efficaces
  sans avoir écrire de traversiers (parcours) d'une manière
  procédurale.

  En étant un langage déclaratif, \emph{Cypher} se concentre sur la
  clarté d'exprimer \textit{quoi retrouver dans un graphe et non
    comment le faire}. Ceci est en contraste aux langages impératifs
  comme Java et aux langages script comme \emph{Gremlin}
  \ref{sec:gremlin}, ce qui rend le fait d'optimisation de requêtes un
  détail d'implémentation non exposé aux utilisateurs. \emph{Cypher}
  est inspiré de plusieurs approches et construit sur des pratiques
  établies pour l'interrogation expressif des bases de données. La
  plupart des mots clés comme \verb|WHERE| et \verb|ORDER BY|, et La
  concordance de patterns sont hérités directement des langages
  déclaratifs comme \emph{SQL} et \emph{SPARQL} \ref{sec:sparql} avec
  quelque propriétés inspirés des langages fonctionnels comme
  \emph{Haskell} et \emph{ML}.

  Le langage \emph{Cypher} comporte un nombre de clauses distinctes,
  des clauses pour l'interrogation du graphe comme:
  \begin{itemize}
  \item [\texttt{MATCH}]: Utilisé pour pour décrire Le pattern du graphe
    à correspondre, principalement sur la base de relations entre les
    nœuds du graphe.
  \item [\texttt{WHERE}]: Sert à un critère de filtrage.
  \item [\texttt{RETURN}]: Spécifie ce qu'il faut retourner comme
    résultat finale de requête.
  \end{itemize}

  \emph{Cypher} contient en outre des clauses pour l'écriture, la mise
  à jour et suppression de données, par exemple:

  \begin{itemize}
  \item [\texttt{CREATE}]: Crée des nœuds ou des relations.
  \item [\texttt{SET}]: Affecte des valeurs aux propriétés..
  \item [\texttt{DELETE}]: supprime des noeuds, relations ou propriétés.
  \end{itemize}

  \newpage
  \section{Conclusion}

%%% Local Variables:
%%% mode: latex
%%% TeX-master: "../main"
%%% End:
