% !TEX root = ../main.tex
\chapter{Les services web: Vue d'ensemble}
Ce chapitre établit une étude du fondement théorique de notre travail
à savoir les concepts de base du paradigme service Web.  Nous
commençons d'abord par présenter un tour d'horizon définissant
l'architecture de référence de ce paradigme ainsi que quelque
définitions proposées dans la littérature. Ensuite nous nous
intéressons à montrer les limitation de l'approche syntaxique de la
description des services web et l'apport de l'enrichissement
sémantique de cette dernière aux processus de la découverte et la
composition des services Web.

\newpage
\section{Définitions et caractéristiques}
\label{sec:ws-notions-de-base}

Les services Web est l'approche la plus populaire pour mettre en œuvre
une architecture SOA . Le terme service Web qualifie tout service
disponible par Internet, qui utilise un format standard (généralement
\acrshort{xml} ou \acrshort{json}) d'échange de messages, et qui n'est
pas lié à un système d'exploitation ou un langage de programmation
particulier.

\label{sec:ws-definition}
Curbera et al. \cite{curbera2001web} a définissent un service web
comme \emph{``une application réseau capable d'interagir par le moyen
  des standards et des protocoles via des interfaces bien spécifiés,
  dans lequel est décris utilisant un langage de description
  fonctionnel standardisé''}.

Les services Web ont été proposés initialement par IBM
\cite{kreger2001web} et Microsoft, puis en standardisés par le
\acrshort{w3c} \footnote{\url{http://www.w3.org/}} et définis
\cite{WSA} par :

\emph{``Un service web est un système conçu pour permettre
  d'interopérabilité des applications à travers un réseau.  Il est
  caractérisé par un format de description
  interprétable/compréhensible automatiquement par la machine,
  D'autres systèmes peuvent interagir avec le Service Web selon la
  manière prescrite dans sa description et en utilisant des messages
  SOAP, généralement transmis via le protocole HTTP et sérialisés en
  XML et en d'autres standards du Web ''}.

Cette définition surligne les caractéristiques clés de services Web
\cite{fremantle2002enterprise}:

\renewcommand{\descriptionlabel}[1]{\hspace{1cm}\textbullet~\textsf{#1}}
\begin{description}
\item[Basés sur des protocoles Internet] : L'utilisation de
  \acrshort{http} pour le transport des informations permet de
  traverser les contrôles d'accès dans un environnement hétérogène.

\item[Interopérables] : Le standard \textsc{SOAP} \cite{box2000simple}
  définit comme étant un protocole destiné à l'échange de messages
  structurés véhiculé généralement sur \textsc{HTTP} et sérialisé en
  \textsc{XML}, permettant le support pour l'interopérabilité.

\item[Basés sur XML] : Le méta-langage de balisage \textsc{XML}
  \textit{eXtensible Markup Language} est un standard Web ouvert par
  \textsc{W3C} \cite{bray1998extensible} offre un cadre standard pour
  la définition de documents Interprétable par des machines.
\end{description}

M. P. Papazoglou \cite{papazoglou2003service} de apporte une
autre définition des services web:\\ \emph{``Les services Web sont
des éléments auto-descriptifs et indépendants des plateformes
permettent la composition faible coût d’applications
distribuées. Les services Web effectuent des fonctions allant de
simples requêtes des processus métiers complexes. Les services Web
permettent aux organisations d’exposer leurs programmes résultats
sur Internet (ou sur un intranet) en utilisant des langages (basés
sur XML) et des protocoles standardisés et de les mettre en œuvre
via une interface auto-descriptive basée sur des formats
standardisés et ouverts''}

% TODO: make a comment on this def and introduce the web services %
% composition idea:

% \subsection{L'évolution des styles des services web}
% \label{sec:levol-des-styl}
% \input{content/evolution.tex}
% \newpage

\section{L'architecture de référence et technlogies associées}
\label{sec:reference-arch}
Cette architecture a été proposée afin de promouvoir
l'interopérabilité et l'extensibilité des services Web dans
l'ensemble, une architecture de base des services Web est constitué
d'un fournisseur du servicee \textit{(Provider)}, un annuaire des
services \textit{(Service Registry)}, et un client du service
\textit{(Service Requester)}. La figure \ref{fig:ws_roles} montre
comment ces trois rôles interagissent.

%!TEX root = ../main.tex
\begin{figure}[h]
    \centering 
    \includegraphics[width=1\textwidth]{figs/ws_roles.eps}
    \caption{Architecture de référence des services Web \cite{gottschalk2002introduction}}
    \label{fig:ws_roles}
\end{figure}

%%% Local Variables:
%%% mode: latex
%%% TeX-master: "../main.tex"
%%% End:

\renewcommand{\descriptionlabel}[1]{\hspace{1cm}\textbullet~\textsf{#1}}
\begin{description}
\item[Le fournisseur]: Un prestataire des services fournit l'interface
  pour le service Web et l'implémentation de l'application.

\item[L'annuaire]: Un registre de service est une façon dont les
  services Web sont officiellement publiés. Le registre de service est
  basée sur la spécification \textsc{UDDI} et reflète des informations
  sur les services fournis par le fournisseur de services.

\item[Le client]: Un client d'un service est le consommateur d'un
  service Web, il utilise le registre de service pour obtenir des
  informations et pour pouvoir accèder à un service Web.
\end{description}

Pour qu'une application profite des services Web, trois comportements
doivent avoir lieu : \textit{la publication} des descriptions du
service, \textit{la découverte} de ces descriptions et enfin
\textit{l'invocation} des services.

\renewcommand{\descriptionlabel}[1]{\hspace{1cm}\textbullet~\textsf{#1}}
\begin{description}
\item[Publier]: Pour être accessible, une description de service doit
  être publiée afin que le client puisse la trouver.

\item[Découvrir]: Lors de la découverte d'un service , le client
  interroge un annuaire (ou un registre) des services pour une
  catégorie requise des services et récupère une description
  détaillée.

\item[Invoquer (bind)]: le client peut ainsi invoquer une
  fonctionalité particulière dans le service cible utilisant les
  détails de liaison dans la description du service pour le localiser,
  contacter et l'appeler.
\end{description}

Les services Web sont construits autour de standards qui sont
\acrshort{soap}, \acrshort{wsdl} et \acrshort{uddi} assurant
respectivement leur communication, leur description et leur
découverte.

  \subsection{Communication: SOAP}
  \label{sec:soap}
  Développé par IBM\footnote{\url{http://www.ibm.com}} et
  Microsoft\footnote{\url{http://www.microsoft.com}}
  \cite{box2000simple}, Le langage \textsc{SOAP} est une
  recommandation \textsc{W3C} \cite{mitra2003soap} qui le définit
  comme étant un protocole destiné à l'échange de messages structurés,
  permettant d'invoquer des applications sur des réseaux distribués.

  \textsc{SOAP} est basé sur \textsc{XML} pour mettre en place un
  mécanisme valable d'échange des données indépendant du modèle de
  programmation de l'application et du système d'exploitation.

  Un message \textsc{SOAP} est un document XML constitué d'une
  enveloppe \textsc{SOAP} obligatoire, d'un en-tête \textsc{SOAP}
  facultatif et d'un corps \textsc{SOAP} obligatoire (voir la figure
  \ref{fig:soap-message-structure}):

  \begin{figure}[h]
    \centering
    \includegraphics[width=0.5\textwidth]{figs/soap_structure.eps}
    \caption{ Les éléments d'un message \textsc{SOAP}}
    \label{fig:soap-message-structure}
\end{figure}


  \renewcommand{\descriptionlabel}[1]{\hspace{1cm}\textbullet~\textsf{#1}}
  \begin{description}
  \item[Enveloppe \texttt{<Envelope>}]: L'élément racine du message
    \textsc{SOAP}, définissant le contexte du message, son
    destinataire et son contenu, il englobe l'en-tête et le corps.

  \item[En-tête \texttt{<Header>}]: Un mécanisme générique permettant
    d'ajouter des fonctions à un message \textsc{SOAP} d'une manière
    modulaire sans accord préalable entre les parties en
    communication.  Des exemples d'extension qui peuvent être
    implémentées comme des en-têtes sont des authentifications, des
    transactions, des paiements

  \item[Corps \texttt{<Body>}]: Contient les informations obligatoires
    destinées à l'ultime destinataire du message, il sert comme un
    container pour les informations mandataires à l'intention du
    récepteur du message. \textsc{SOAP} définit un élément pour le
    corps, qui est l'élément \texttt{<Fault>} (Erreur) utilisé pour
    rapporter les erreurs.
  \end{description}

  \subsection{Description: WSDL}
  \label{sec:wsdl}
  Le langage de description des services Web \acrshort{wsdl}
  \cite{christensen2001web, chinnici2007web} est une recommandation du
  \acrshort{w3c}, maintenant dans sa deuxième version.  \textsc{WSDL}
  est basé sur \textsc{XML} pour décrire les fonctions opérationnelles
  de services Web. La description \textsc{WSDL} est composée d'un
  interface et des implémentations. L'interface est une définition
  abstraite et réutilisable service qui peut être référencée par
  plusieurs implémentations.

  %!TEX root = ../main.tex
\begin{figure}[h]
    \centering
    \includegraphics[width=0.7\textwidth]{figs/wsdl-document-structure.eps}
    \caption{Structure d'un message \textsc{WSDL}.}
    \label{fig:wsdl-document-structure}
\end{figure}

%%% Local Variables:
%%% mode: latex
%%% TeX-master: "../main"
%%% End:


  Le langage de description de services Web \acrshort{wsdl}
  \cite{chinnici2007web} fournit un modèle ainsi qu'un langage basé
  sur \textsc{XML} de description de services Web. Un fichier
  \textsc{WSDL} comprend une description des fonctionnalités d'un
  service, mais il ne se préoccupe pas de l'implantation de celles-ci.
  Il contient aussi des informations concernant la localisation du
  service, ainsi que les données et les protocoles à utiliser pour
  l'invoquer. En pratique, le document \textsc{WSDL}
  \footnote{\url{http://www.w3.org/TR/wsdl20/}} est un document
  \textsc{XML} qui se divise en deux parties \cite{elie2010} :

  \SpecialItem
  \begin{itemize}
  \item La définition \textbf{abstraite} de l'interface du service
    avec les opérations supportées par le service Web, ainsi que leurs
    paramètres et les types des données.

  \item La définition \textbf{concrète} de l'accès au service avec la
    localisation, par une adresse réseau du fournisseur de service
    \footnote{Service Endpoint}, et les protocoles spécifiques
    d'accès.
  \end{itemize}

  La partie abstraite d'un document \textsc{WSDL} contient deux
  sous-parties:

  \SpecialItem
  \renewcommand{\descriptionlabel}[1]{\hspace{1.5cm}\texttt{#1}}
  \begin{description}
  \item[<Types>]: décrit sous la forme d'un schéma \textsc{XML} les
    types des données échangées entre le client et le fournisseur de
    services \cite{part20012}.

  \item[<Interface>]: Les interfaces\textsc{WDSL} offrent une manière
    abstraite de décrire la fonctionnalité du service, ils ont défini
    les opérations (éléments \texttt{<operation>}) en terme de
    paramètres d'entrée et de sortie sous frorme d'un un modèle
    d'échange de message \textit{(message exchange
      pattern)}. \textsc{WSDL} contient huit modèles de messages
    prédéfinis, mais on peut facilement définir de nouveaux.
  \end{description}

  La définition concrète d'un document \textsc{WSDL} est constituée
  de:

  \SpecialItem
  \renewcommand{\descriptionlabel}[1]{\hspace{1.5cm}\texttt{#1}}
  \begin{description}
  \item[<Binding>]: L'élément \texttt{Binding} reprend les opérations
    de l'élément \texttt{<Interface>} et leurs associe un protocole de
    transfert et des spécifications des formats de données de message.

  \item[<Service>]: Cet élément définit la localisation du service Web
    décrit. Pour chaque interface décrite, un élément service lui est
    associé. Le sous-élément \texttt{<endpoint>} définit un port
    d’accès en référençant l'élément \texttt{<binding>} associé et en
    déclarant l'\textsc{URL} localisant le service (avec l'attribut
    \texttt{<address>}).
  \end{description}
  % parler sur la limitation de la description syntaxique de WSDL,
  % réferencer les \ref{sec:ws-description}
  \newpage
  \subsection{Découverte: UDDI}
  \label{sec:uddi}
  %TODO
  \acrshort{uddi} \cite{clement2004uddi} est une standardisation pour
  la publication et la découverte des services Web initialement conçue
  et spécifiée par le Consortium de standards
  OASIS\footnote{\url{https://www.oasis-open.org}}, il est le résultat
  d'un accord d'un ensemble d'industriels
  Ariba\footnote{\url{http://www.ariba.com/}}, IBM, Microsoft, etc en
  vue de devenir le registre standard de la technologie des services
  Web.

  \textsc{UDDI} complète les technologies basiques de services Web en
  permettant de créer un \textbf{annuaire} permettant de localiser sur
  le réseau le services web recherchés, les services référencés dans
  \textsc{UDDI} sont accessibles par l'intermédiaire du protocole de
  communication \textsc{SOAP}, et la publication des informations
  concernant les fournisseurs et les services doit être spécifiée en
  \textsc{XML} afin que la recherche et l'utilisation soient faites de
  manière \textbf{dynamique} et \textbf{automatique}.

  Un \textsc{UDDI} peut appartenir à un domaine public comme internet
  ou tout autre réseau accessible à un nombre non limité
  d'utilisateurs, comme il peut appartenir à un domaine restreint
  comme l'intranet d'une entreprise ou d'un groupe d'entreprise.

  Les données stockés dans l'\textsc{UDDI} sont structurées (en
  \textsc{XML}) et organisées en trois parties connues:

  \SpecialItem
  \begin{description}
    \item[Pages blanches]: fournissent des descriptions générales sur
      les fournisseurs de services à savoir le nom de l'entreprise qui
      fournit le service, son identificateur commercial, ses adresses,
      etc.

    \item[Pages jaunes]: comportent des descriptions détaillées sur
      les fournisseurs de services catalogués dans les pages blanches
      d'une de façon taxonomique (selon secteurs d'activités par
      exemple).

    \item[Pages vertes]: fournissent des informations techniques sur
      les services Web catalogués. Ces informations incluent la
      description du service, les adresses \textsc{URL}, du processus
      de son utilisation et des protocoles utilisés pour son
      invocation.
  \end{description}

  % TODO: talk about OWL-S and why UDDI approach is semantically poor
  % TODO : Conclure cette subsection par la mention du problème de la
  % découverte automatique des services Web et l'insuffisance de la
  % description syntaxique.
  % \newpage

  % critiquer la découvert manual des services dans UDDI
  % référencer \ref{sec:ws-localisation}

\section{Description des services web}
\label{sec:ws-description}

% Un service Web consiste à la définition d'une entité logiciel
% modulaire, \textit{auto-descriptive} et \textit{autonome}
% \textit{accessible} via Internet \cite{curbera2001web}. Dans la
% section précédente, nous avons vu l'architecture de base des services
% Web et la pile protocolaire pour la mise en place d'une telle
% architecture.

Une description du service Web est un document par lequel le
fournisseur de services communique au client les spécifications pour
invoquer le service Web \cite{lopez2008selection}. Malgré les
améliorations apportées au standard \textsc{WSDL} dans son deuxième
version \cite{chinnici2007web}, la description du service reste
uniquement au niveau fonctionnel, c'est-à-dire qu'elle contient la
manière dont on peut utiliser le service et non ce que fait le
service, le standard \textsc{WSDL} est limité à l'énumération des
opérations et à la description des types des paramètres d'entrée et de
sortie associés, elle ne caractérise pas la sémantique de la
fonctionnalité accomplie par le service.

Par conséquent, la description \textsc{WSDL} reste insuffisante lors
des processus de sélection, découvert et de la composition. Pour
pallier cette Difficulté, plusieurs approches
\cite{sivashanmugam2003adding,mcilraith2001semantic,
  mcilraith2003bringing, fensel2002web} proposent de rajouter une
couche sémantique au dessus de \textsc{WSDL} complétant la description
syntaxique par des précisions sémantiques.

%!TEX root = ../main.tex
\begin{figure}[h]
    \centering
    \includegraphics[width=1\textwidth]{figs/3w_to_sws.eps}
    %TODO translate
    \caption{Web evolution to Semantic Web services \cite{fensel2002semantic}.}
    \label{fig:3w_to_sws}
\end{figure}

%%% Local Variables:
%%% mode: latex
%%% TeX-master: "../main"
%%% End:
\newpage
Les services Web enrichis par des métadonnées supplémentaires
exprimant leur la sémantique sont appelés \textit{les services Web
sémantiques} \cite{fensel2002semantic, mcilraith2001semantic}. Les
services de Web sémantique sont le résultat de l'évolution Web dans
deux directions \cite{bartalos2011effective} (illustré dans La
figure\ref{fig:3w_to_sws}):

\begin{enumerate}
  \item L'interopérabilité dynamique des éléments sur le Web.
  \item L'amélioration de la description syntaxique services Web
\end{enumerate}

Dans cette section nous allons présenter les diverses approches
sémantiques visant à préciser la description d'un service en insistant
sur les approches d'annotation sémantique et sur les ontologies de
services.

La notion du Web sémantique est abordée brièvement dans l'annexe
\ref{annexe:semantic-web}.

  \subsection{Annotations sémantiques}
  \label{sec:semantic-annot}

  L'annotation sémantique consiste à enrichir et à compléter la
  description d'un service. Elle établit des correspondances entre des
  éléments de la description et des concepts d'un ensemble
  d'ontologies de références. Une ontologie de référence permet de
  représenter un domaine par des structures interprétables par une
  machine. Deux modèles principaux suivent l'approche d'annotation
  sémantique, à savoir \textsc{WSDL-S} et \textsc{SAWSDL}
  \cite{elie2010}.

    \subsubsection{WSDL-S}
    \textsc{WSDL-S} \cite{akkiraju2005web} est le résultat d'un
    travail collaboratif entre IBM, laboratoire LSDSI et l'iniversité
    de Geogia \footnote{\url{http://www.uga.edu/}}.  La spécification
    a devenue une recommandation \textsc{W3C} depuis 2005. Son
    objectif principal est de fournir un processus d'annotation
    sémantique compatible avec les technologies
    existantes. Pratiquement, Le méta-modèle \textsc{WSDL-S} repose
    sur les capabilités du modèle \textsc{WSDL} en rajoutant trois
    éléments majeurs \texttt{<category>}, \texttt{<effect>} et deux
    attributs \texttt{modelReference} et \texttt{schemaMapping}. Les
    éléments introduits permettent de rajouter des informations qui
    n'étaient pas prises en compte dans \textsc{WSDL} comme \emph{les
      préconditions} et \emph{les effets} d'une opération. Tandis que
    les attributs permettent de référencer des concepts dans des
    ontologies de référence, ces préconditions et effets ensemble avec
    les annotations sémantiques des éléments \texttt{<inputs>} et
    \texttt{<outputs>} permet de l'automatisation du processus de
    découvert de services.

    \subsubsection{SAWSDL}
    La spécification \acrshort{sawsdl} \cite{kopecky2007sawsdl} est la
    suite de \textsc{WSDL-S} et il partage les mêmes principes de ce
    dernier. issue d'initiative du groupe de travail d'annotations
    sémantiques pour \textsc{WSDL}
    \footnote{\url{http://www.w3.org/TR/sawsdl/}} et soumise au
    \textsc{W3C} en 2007, \textsc{SAWSDL} définit un mécanisme
    d'annoter sémantiquement les interfaces et les opérations
    \textsc{WSDL}, ainsi que les types \textsc{XML schema} en les
    reliant à des concepts dans une ontologie. Cette annotation repose
    sur la définition d'attributs étendant le standard de
    description. Les annotations sémantiques référencent des
    ontologies pré-existantes. Le mécanisme d'annotation de
    \textsc{SAWSDL} est indépendant de tout langage de représentation
    \cite{lopez2008selection} d'ontologies.

    \textsc{SAWSDL} propose deux sortes d'annotations sémantiques: une
    pour identifier le concept sémantique (représentée par l'attribut
    \texttt{modelReference}) et une autre pour faire le lien entre le
    concept et le document \textsc{WSDL} (représentée par les
    attributs \texttt{liftingSchemaMapping} et
    \texttt{loweringSchemaMapping}).

  \subsection{Ontologies de services}
  \label{sec:ont-serices}

  Une ontologie de services saisit les différents aspects liés à la
  description des services et leur utilisation à travers un ensemble
  de concepts, de propriétés et de relations entre eux. Deux modèles
  d'ontologies de services sont décrits dans cette sous-section
  \textsc{OWL-S} et \textsc{WSMO} \cite{elie2010}.

    \subsubsection{WSMO}
    \label{sec:wsmo}

    \acrshort{wsmo} \cite{de2005web} est une modèle conceptuel basé
    sur la spécification \acrshort{wsmf} \cite{fensel2002web} (voir
    \ref{sec:wsmf}) pour la description des divers aspects liés aux
    services Web sémantique. Le but de \textsc{WSMO} est d'automatiser
    le cycle de vie des services web (publication, sélection,
    découverte, composition, etc.) afin de résoudre le problème de
    l'intégration des services Web en définissant une technologie
    cohérente pour description des services Web sémantique.

    Pour formaliser \textsc{WSMO}, le groupe de travail mis au point
    le langage de modélisation \acrshort{wsml} \cite{de2006web} et a
    défini plusieurs variantes de celui-ci, chacun basé sur différents
    formalismes.

    \textsc{WSMO} est constitué de quatre composants: les services
    web, les buts, les ontologies et les médiateurs. Ce modèle permet
    de réaliser un couplage faible entre les services web en utilisant
    un ensemble de médiateurs. Ces derniers assurent les tâches
    d‘intégration d'ontologies, de découverte des services, de
    composition, etc.

    \subsubsection{OWL-S}
    \label{sec:owl-s-1}

  % TODO: elaborate
  \subsection{L'aspect fonctionnel et non fonctionel de description}
  \label{sec:func-vs-non-func}
  % TODO
  {\color{red}
    \cite{el2014cbr4wsd}
  }

  Il existe deux aspects de description des services Web:

    \subsubsection{L'aspect fonctionnel}
    \label{sec:aspect-fonctionnel}

    \subsubsection{L'aspect non fonctionnel}
    \label{sec:aspect-non-fonctionel}

\newpage
\section{Découverte des services web}
\label{sec:ws-discovery}
La découverte de services Web présente un axe de recherche
important. Divers mécanismes de découverte ont été proposés dans la
littérature et plusieurs définitions sont attribuées à ce
concept. Booth \textit{el al.}  décrivent le processus de découverte
comme étant l'acte de \textit{``localisation d'une description
  compréhensible par la machine d'un service éventuellement inconnu au
  préalable écrivant certains critères fonctionnels ''}
\cite{booth2004web}.

% TODO: refactore this paragraph
L'opération de découverte d'un service web est établie essentiellement
en deux phases, nous sélectionnons d'abord la ou les catégories de
services web les plus proches à notre requête, ensuite nous comparons
chacun de leurs membres avec le besoin spécifié par L'utilisateur.

Plusieurs critères peuvent être utilisés pour catégoriser les
approches de découverte \cite{elie2010}:

\begin{itemize}
\item le localisation des services (centralisation/décentralisation
  des annuaires).
\item Le degré d'automatisation du processus de la découverte.
\item le principe de l'algorithme de matching \ref{sec:ws-matching}.
\end{itemize}


{\color{red} %TODO: decouple from \cite{elie2010}
  Dans cette section nous présentons les différentes approches de
  localisation de services basées sur des modèles centralisés ou des
  modèles distribués. Ensuite nous classifions les approches de
  découverte selon le degré d'automatisation du traitement des
  données. \cite{elie2010}
}

Le dernier critère de classification sera présenté dans la prochaine
section \ref{sec:ws-matching}.

% \newpage
\subsection{Localisation des services}
\label{sec:ws-localisation}

  La découverte consiste principalement à localiser les descriptions
  de services répondant à une requête client. Les approches de
  localisation des services récurrentes dans la littérature sont
  classées en deux catégories, à savoir les approches centralisées et
  les approches décentralisées ou distribuées
  \cite{garofalakis2004web}.

    \subsubsection{Approches centralisées}
    \label{sec:central-disc}

    Les premières versions d'\textsc{UDDI} \cite{clement2004uddi}
    (présenté dans \ref{sec:uddi}) reposent sur une approche
    \textit{centralisée} de publication et découverte de services
    Web. le registre \textsc{UDDI} définit un modèle de représentation
    des données et des méta-données nécessaires à la publication,
    reposent sur \textit{un seul annuaire} qui peut être géré par un
    module de mise en correspondances \textit{(matchmaker)}.

    Les approches centralisées de localisation des services Web ne se
    limitent pas à la recherche syntaxique par mots clés. Dans
    \cite{srinivasan2004adding} les auteurs étendent le modèle
    \textsc{UDDI} pour prendre en compte des descriptions
    \textsc{OWL-S} et présentent \textit{OWL-S macthmaker}. L'objectif
    est de permettre le stockage des fonctionnalités des services
    désignées par \textit{capabilities} afin de rendre la découverte
    plus précise.Cette approche bien que riches émantiquement ne
    traite pas l'aspect automatisation de la découverte et repose sur
    un \textsc{UDDI} centralisé.

    \subsubsection{Approcehs décentralisées}
    \label{sec:decentr-disc}

    Les approches décentralisées de découverte des services consistent
    à mettre en place une \textit{fédération} d'annuaires
    \textsc{UDDI} agissant comme une couche d'abstraction reliant
    plusieurs instances d'annuaires \cite {rompothong2003query,
      sivashanmugam2004discovery}.

    Plusieurs travaux dans la littérature proposent des solutions
    décentralisées du processus de publication et découverte. Les
    systèmes \cite{paolucci2003using, schmidt2004peer,
      verma2005meteor} appuient sur des infrastructures pair-à-pair
    \textit{peer-to-peer} et des ontologies pour la publication et la
    découverte des services Web sémantiques permettant une forte
    collaboration entre plusieurs instances d'annuaires \textsc{UDDI}.

    La dernière version d'\textsc{UDDI} \cite{oasis2005specification}
    (version \textit{3.0.2}) reprend le principe d'une fédération
    d'annuaires. Elle décrit un annuaire \textsc{UDDI} comme un
    ensemble de nœuds \textit{(UDDI nodes)} tel que chaque nœud fait
    partie d'un seul annuaire et possède une copie répliquée de schéma
    globale de la fédération. Les nœuds d'un annuaire collaborent pour
    gérer un ensemble de structures de données \textsc{UDDI} et
    permettre une meilleure gestion des requêtes.

  \subsection{Découverte manuelle/automatique}
  \label{sec:ws-desc:manual-vs-auto}

  Selon le richesse sémantique de la description des services, le
  degré d'intervention de l'utilisateur dans la découverte
  variée. Nous pouvons distinguer les approches manuelles et
  semi-automatique d'une part, et les approches automatiques d'une
  autre part \cite{elie2010,garofalakis2004web}.

    \subsubsection{Approches manuelles et semi-automatiques}
    \label{ws-desc:manual}

    Dans une découverte manuelle, le client utilise un service de
    découverte pour localiser et sélectionner manuellement une
    description de service qui répond aux certains critères
    fonctionnels. La recherche de descriptions dans cette approche est
    souvent basée sur une simple comparaison syntaxique entre les mots
    clés de la requête et les descriptions disponibles dans le
    registre des services, ce qui nécessite l'intervention de
    l'utilisateur pour vérifier la pertinence et la fiabilité des
    résultats de la recherche et sélectionner le service Web qui
    répond au mieux à ses exigences.

    Verma \textit{el al.} \cite{verma2005meteor} dans le cadre du
    project \textsc{METEOR-S}, l'infrastructure présente un degré
    d'automatisation plus avancé qu'une simple recherche dans un
    annuaire \textsc{UDDI}. En effet, l'enrichissement sémantique des
    fichiers de description des services Web permet une découverte
    semi-automatique et plus précise.

    \subsubsection{Approches automatiques}
    \label{ws-desc:auto}

    La découverte automatique de services qui répond à un besoin donné
    est considéré comme une étape cruciale vers l'intégration
    dynamique et évolutive des services Web. On entend par découverte
    automatique la possibilité de localiser automatiquement un Web
    service qui répond à des besoins particuliers, Différentes
    approches ont été proposées dans la littérature pour réaliser la
    découverte dynamique de services \cite{paolucci2002semantic,
      bernstein2002discovering, mandell2003bottom,
      benatallah2005automating,keller2005automatic}.

    Dans \cite{mandell2003bottom} les auteurs présentent une approche
    d'automatisation de la découverte des services basée sur la
    technologie \acrshort{bpel} (voir \ref{sec:bpel}) afin de
    permettre une composition plus dynamique. Les auteurs proposent
    d'étendre la description \textsc{BPEL} par l'intégration d'une
    description sémantique \textsc{OWL-S} (voir \ref{sec:owl-s}) de
    type \textit{service profile} et la mise en place d'un service de
    découverte sémantique équipé par un raisonneur automatique. Au
    cours de l'interprétation du document \textsc{BPEL} le service de
    découverte sémantique intervient et envoie des requêtes générées à
    partir des \textit{service profile} du processus \textsc{ BPEL}
    vers un annuaire contenant des services sémantiquement décrits en
    \textsc{OWL-S}.
    %% TODO decouple from \cite{elie2010} words
    Keller \textit{et al} \cite{keller2005automatic} étudient un
    modèle de localisation automatique de services Web décrits via le
    modèle \textsc{WSMO}. Le modèle proposé traite à la fois l'aspect
    statique et l'aspect dynamique de la description des services
    Web. L'aspect statique consiste à trouver les correspondances
    entre les descriptions des services et une requête
    utilisateur. L'aspect dynamique raffine le choix des services
    pré-sélectionnés en interrogeant le fournisseur pour vérifier
    certaines critères non fonctionnels.

\section{Matching des services Web}
\label{sec:ws-matching}
Dans le contexte d'une découverte de services, le \textit{Matching}
(ou le \textit{Matchmaking}) est définie comme un processus qui
nécessite un annuaire des services de prendre une requête en entrée,
et de revenir tous les services qui peuvent satisfaire les exigences
spécifiées dans la requête d'entrée \cite{li2004software}. Cette
opération nécessite la recherche de similarités indiquant le degré de
rapprochement entre les paramètres descriptives fonctionnels et
non fonctionnels (~\ref{sec:func-vs-non-func}) des services référencés
d'une côté, et les paramètres de la requête client d'une autre.

Le calcul de similarité peut être basé sur des données syntaxiques ou
sémantiques plus expressives \cite{elie2010}. Dans la suite nous
résumons les approches récurrentes du \textit{Matching} des services
Web rencontrées dans la littérature.

  \subsection{Matching syntaxique}
  \label{sec:matching-syntactique}
  {\color{red} Le principe général des approches basées sur la syntaxe
    des descriptions des services est la comparaison syntaxique entre
    la requête, basée mots clés du client et les descriptions
    syntaxiques (\textsc{WSDL}) des services Web
    référencés. \cite{chelbabi2012decouverte} 
    
    référence ver \ref{sec:syntactic-sim}
  }  


  {\color{red} 
    Pour pallier cette déficience, les capacités d’extensions du
    modèle de données UDDI sont utilisées pour mettre au point une
    correspondance (ou « mapping ») entre les déclarations.
  }

  \subsection{Matching sémanique}
  \label{sec:matching-semanique}  
  {\color{red} De récents travaux se sont focalisés sur la description
    sémantique des services Web. Ce développement est de plus en plus
    significatif puisqu'il semble pouvoir aborder certaines
    insuffisances des approches basées sur les mots clés. Les
    ontologies sont le modèle utilisé pour la représentation
    sémantique des services Web, elle permet d’établir des relations
    sémantiques entre les différents concepts d'un
    domaine. \cite{chelbabi2012decouverte}

    Consiste à chercher des correspondances sémantiques à partir des
    concepts, et non pas des étiquettes. Il ne suffit pas d'examiner
    la syntaxe des mots, mais leurs significations ou leurs
    interprétations dans le domaine d'application. Les ontologies de
    domaine et les techniques basées modèle sont utilisées dans ce
    type de matching.

    Les approches de calcul de similarités sémantique reposent sur des
    modèles de description sémantique. Par la suite nous résumons
    les deux approches majeures, basées respectivement sur le modèle
    \textsc{OWL-S} et sur le modèle \textsc{WSMO} \cite{elie2010}.

    référence ver \ref{sec:semantic-sim}
  }  

    \subsubsection{Matching sémantique basé sur WSMO}
    \label{sec:match-wsmo}
    \cite{paolucci2002semantic, keller2004wsmo}

    \subsubsection{Matching sémantique basé sur OWL-S}
    \label{sec:match-owls}    
    \cite{paolucci2002semantic,benatallah2003request,
        benatallah2005automating, martin2004owl}

  % \begin{mydef}[Matching]
  %   c'est le processus de découverte des liaisons et des
  %   correspondances entre les entités de différentes représentations à
  %   travers un algorithme de matching.
  % \end{mydef}
  % (\cite{lecue2006formal}, \cite{paolucci2002semantic},
  % \cite{li2004software})\\

  % Le matching, l'alignement, le mapping...ect. Sont tous des termes
  % utilisées pour référencer le concept de recherche de
  % similarité. Voici quelques définitions de ces termes utilisés.

  % Matchmaking is defined as a process that requires a repository host
  % to take a query or an advertisement as input, and to return all the
  % advertisements that may satisfy the requirements specified in the
  % input query or advertisement.

  % Dans le contexte d'une architecture à services, le
  % \textit{Matchmaking} (ou le \textit{Matching}) est définie comme un
  % processus qui nécessite un annuaire des services de prendre une
  % requête en entrée, et de revenir tous les services qui peuvent
  % satisfaire les exigences spécifiées dans la requête d'entrée
  % \cite{li2004software}. Formellement, Le processus de
  % \textit{Matchmacking} peut se spécifié comme de suit:

  % Soit $\Phi$ est l'ensemble de toutes les services référencés dans un
  % annuaire des services donné. Pour une requête \verb|R|.

  % \begin{mydef}[Alignement]
  %   on parle souvent de l'alignement des ontologies. C'est un ensemble
  %   de correspondances entre deux ou plusieurs
  %   ontologies. L'alignement est la sortie d’un processus de matching
  % \end{mydef}

  % \begin{mydef}[Mapping]
  %   chercher des correspondances pour établir des transformations
  %   entre deux objets de même nature mais pas de même forme. Par
  %   conséquent, le mapping utilise les résultats du matching pour
  %   effectuer les transformations des objets.
  % \end{mydef}

  % Comme on le voit, tous ces termes prennent bien en compte la notion
  % de recherche de correspondance entre des concepts, dans certains
  % contextes ils sont utilisés indifféremment.  Pour les services web,
  % on parlera du matchnig ou matchmaking pour exprimer la mise en
  % correspondance entre deux entités, et du mapping pour exprimer la
  % transformation ou conversion des types de données.

  % Nous distinguons deux techniques de matching : matching syntaxique
  % et matching sémantique . Ils diffèrent dans la façon dont sont
  % calculés les éléments correspondants et sur le type de relation de
  % similitude \verb|M| utilisée:

  %   \subsubsection{Matching syntaxique}
  %   \label{sec:matching-syntactique}
  %   Utilisé pour l'extraction automatique de motifs textuels écrits en
  %   langage naturel en utilisant la méthode d'étiquetage, différentes
  %   techniques sont utilisées :
  %   \renewcommand{\descriptionlabel}[1]{\hspace{0.1cm}\textbullet~\textsf{#1}}
  %   \begin{description}
  %   \item [Techniques basées string (à mots)]: s méthodes prennent
  %     l’avantage de la structure des mots qui est considérée comme
  %     étant une séquence de lettres. Elles devraient, par exemple,
  %     reconnaitre une similarité entre Book et textBook. Certaines
  %     fonctions de calcul de distance convertissent une paire de
  %     chaine de caractères en un nombre réel. Si le nombre est petit
  %     (par rapport à un seuil) cela indique une grande similarité
  %     entre ces deux chaines de caractères (ex : distance de Hamming).

  %   \item [Techniques basées langage]: Ces méthodes considère les noms
  %     ou bien un terme comme une séquence de mots dans un langage
  %     naturel (peer-review, reviewed-paper).  Ces méthodes utilisent
  %     les techniques de traitement du langage naturel pour reconnaitre
  %     les similarités entre les mots d’un terme par exemple : les
  %     techniques de tokenization (segmentation), lemmatisation...etc.
  %     Techniques utilisant des ressources linguistiques: comme les
  %     thesaurus sont utilisées pour faire correspondre des mots en se
  %     basant sur les relations linguistiques entre eux (synonymes,
  %     hyponyme). Par exemple dans le cas du matching entre les
  %     services web on fait correspondre les noms d’opérations et les
  %     noms des paramètres qui sont considérés comme des mots de
  %     langage naturel.
  %   \end{description}

  %   \subsubsection{Matching sémanique}
  %   \label{sec:matching-semanique}
  %   Consiste à chercher des correspondances sémantiques à partir des
  %   concepts, et non pas des étiquettes. Il ne suffit pas d'examiner
  %   la syntaxe des mots, mais leurs significations ou leurs
  %   interprétations dans le domaine d’application. Les ontologies de
  %   domaine et les techniques basées modèle sont utilisées dans ce
  %   type de matching.

  %   \begin{description}
  %   \item [Ontologie de domaine spécifique]: spécifique peut être
  %     utilisée comme source externe. Elles se focalisent sur un
  %     domaine particulier et utilisent des termes qui sont propres à
  %     ce domaine et qui ne sont pas liés aux concepts similaires dans
  %     d’autres domaines. Par exemple dans le domaine de transport, on
  %     trouve : bus, avion, taxi...etc. On peut trouver des termes qui
  %     sont syntaxiquement identiques mais possédant des
  %     interprétations différentes dans des domaines différents.

  %   \item [Techniques basées modèle]: Appelées aussi les méthodes de
  %     déduction, les algorithmes manipulent les entrées basées sur
  %     leurs interprétations sémantiques. Si deux entités sont
  %     similaires alors ils partagent les mêmes interprétations. Parmi
  %     ces méthodes on trouve les techniques de satisfiabilité
  %     propositionnelle [Giu, 2003A] Dans le cas du matching des
  %     services web, ces derniers sont décrits à l’aide de supports
  %   \end{description}

  %   Dans le cas du matching des services web, ces derniers sont décrits
  %   à l'aide de supports de description sémantique (OWL-S par
  %   exemple). Ces descriptions contiennent des informations clés pour
  %   leur mise en correspondance (les paramètres fonctionnels et non
  %   fonctionnels).

\newpage
\section{Conclusion}
% faire un petit récapitulatif sur les technologies des services web
% rappeler de notre problème principale : composition des services web

%%% Local Variables:
%%% mode: latex
%%% TeX-master: "../main"
%%% End:
