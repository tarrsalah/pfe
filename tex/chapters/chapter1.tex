%% !TEX root = ../main.tex
\chapter{Services Web}
\label{ch:web-service}

\section*{Introduction}
\addcontentsline{toc}{section}{Introduction} \markboth{INTRODUCTION}{}

Plusieurs paradigmes de développement de logiciels ont été proposés
pour satisfaire le besoin de la \textit{réutilisation}, le paradigme
orienté service (\acrshort{soa}) représente aujourd'hui une approche
largement adoptée pour le développement des systèmes d'information
distribués sur \textit{Internet}. En effet, il adopte les avantages
des autres paradigmes comme l'orientaté objet et l'orienté composant,
tels que \textit{l'encapsulation}, \textit{la modularité}, \textit{le
  couplage faible} et \textit{l'interopérabilité}. Le paradigme
\acrshort{soa} possède plusieurs implémentations telles que
\textit{DCOM}~\cite{frank1997dcom}, \textsc{CORBA}
\cite{vinoski1997corba},\textsc{RPC}~\cite{bloomer1992power} et
\acrshort{rest}~\cite{fielding2000architectural}. Dans ce travail,
nous nous intéressons uniquement aux services Web basés sur la
spécification \acrshort{soap} (\textit{SOAP-based Web services}), qui
représentent la concrétisation la plus répandue dans
l'industrie.\bigskip

Ce chapitre établit une étude du fondement théorique de notre travail
à savoir les notions de base du paradigme service Web. Nous commençons
par introduire le concept de (\acrshort{soc}) et \acrshort{soa}, ainsi
que les services Web~\ref{sec:ws-definitions}, leurs architectures et
leurs infrastructures, puis nous présentons un socle technologique
très répandu pour la mise en œuvre d'une telle
architecture~\ref{sec:ws-standards}. Ensuite nous nous focalisons sur
les modèles de description des services tout en mettant l'accent sur
l'enrichissement sémantique de la représentation des services
Web~\ref{sec:ws-description}. La dernière partie de ce
chapitre~\ref{sec:ws-discovery} est consacrée à la présentation de
quelques approches de découverte des services Web rencontrées dans la
littérature.

\newpage
\section{Définitions}
\label{sec:ws-definitions}
L'objectif de cette section est d'introduire les concepts et
terminologies associés au paradigme \textit{orientaté service}. Dans
la première section~\ref{sec:soc}, nous allons définir le
\textit{``Service oriented computing''} (\acrshort{soc}) comme un
modèle émergent de développement rapide des applications distribuées
dans les environnements hétérogènes. Nous présentons ensuite
l'architecture orientée service pour réaliser le \acrshort{soc}
(section~\ref{sec:soa}). Dans la section \ref{sec:ws-def}, nous nous
focalisons sur les services Web en présentant quelques définitions et
caractéristiques rencontrées dans la littérature.

  \subsection{Service oriented computing}
  \label{sec:soc}
  L'approche orientée service, en anglais \textit{``Service oriented
    computing''} (\acrshort{soc}) est un paradigme de programmation
  qui permet la construction d'applications à partir d'entités
  logicielles particulières, que sont les \emph{services}.\bigskip

  Selon Papazoglou~\cite{papazoglou2003service}: \textit{``Les
    services des éléments logiciels auto-descriptifs, définis
    indépendamment d'une toute plate-forme ou technologie''}\bigskip

  Les fonctionnalités offertes par un service peuvent allant de
  répondre aux requêtes simples à l'encapsulation des processus
  métiers aussi complexes.\medskip

  Selon le style architectural \acrshort{soc}, les applications sont
  construites à partir de services autonomes, auto-descriptifs et
  faiblement couplés, exposant leurs fonctionnalités sous forme
  d'interfaces bien définies en s'abstrayant complètement de leurs
  implémentations. Grâce à leurs descriptions publiées, ces services
  peuvent être recherchés et ensuite invoqués par des utilisateurs ou
  d'autres services clients, permettant une meilleure collaboration
  entre plusieurs entreprises de différents domaines.\medskip

  Une architecture clé pour réaliser l'approche \acrshort{soc} est
  l'architecture orientée services (\acrshort{soa}). Dans le même
  sens que la \acrshort{soc}, la \acrshort{soa} introduit un paradigme
  pour la construction des applications distribuées où les services
  élémentaires peuvent être publiés, découverts et composés afin de
  créer des services plus complexes.

  \subsection{Architecture orientée services}
  \label{sec:soa}
  Pour réaliser le style architectural présenté précédemment
  (\acrshort{soc}), il faut mettre en place un environnement
  d'intégration et d'exécution des services. L'architecture orientée
  services (\acrshort{soa}) a proposé cet environnement afin de
  promouvoir l'interopérabilité et l'extensibilité de services dans
  l'ensemble.\bigskip

  Selon \textit{OASIS}\footnote{\url{https://www.oasis-open.org}}:
  \textit{``L'architecture orientée services est un paradigme
    permettant d'organiser et d'utiliser des fonctionnalités
    distribuées pouvant être de domaines variés. Cela fournit un moyen
    uniforme d'offrir, de découvrir, d'interagir et d'utiliser ces
    fonctionnalités pour produire le résultat désiré avec des
    préconditions et des buts mesurables.''}\bigskip

  Une architecture à base de
  services~\cite{gottschalk2002introduction} est constitué d'un
  fournisseur de service \textit{(service provider)}, un annuaire des
  services \textit{(service registry)}, et un client de service
  \textit{(service requester)}.

  \input{figs/ws-basic-arch.tex}


  La figure \ref{fig:ws-basic-arch} montre comment ces trois rôles
  interagissent:

  % TODO: rewrite this shit!
  \renewcommand{\descriptionlabel}[1]{\hspace{0.5cm}\textbullet~\textsf{#1}}
  \begin{description}
  \item[Le fournisseur]: Un prestataire de services fournit
    l'interface pour le service Web et l'implémentation de
    l'application.

  \item[L'annuaire]: Un registre de service est une façon dont les
    services Web sont officiellement publiés. Le registre de service
    est basée sur la spécification \acrshort{uddi} et reflète des
    informations sur les services fournis par le fournisseur.

  \item[Le client]: Un client d'un service est le consommateur d'un
    service Web, il utilise le registre de service pour obtenir des
    informations et pour pouvoir accéder à un service Web.
  \end{description}
  \enddescription

  Pour qu'une application profite des services Web, trois
  comportements doivent avoir lieu: \textit{la publication} de
  descriptions des services, \textit{la découverte} de ces
  descriptions et enfin \textit{l'invocation} des services:

  \renewcommand{\descriptionlabel}[1]{\hspace{0.5cm}\textbullet~\textsf{#1}}
  \begin{description}
  \item[Publier]: Pour être accessible, une description de service
    doit être publiée afin que le client puisse la trouver.

  \item[Découvrir]: Lors de la découverte d'un service, le client
    interroge un annuaire (ou un registre) de services pour une
    catégorie requise de services et récupère une description
    détaillée.

  \item[Invoquer (bind)]: le client peut ainsi invoquer une
    fonctionnalité particulière dans le service cible utilisant les
    détails de liaison dans la description du service pour le
    localiser, contacter et l'appeler.
  \end{description}
  \enddescription

  La \acrshort{soa} n'est pas liée à une technologie particulière. En
  effet, il existe une multitude d'implémentations techniques de cette
  architecture, y compris les services Web~\cite{WSA},
  \acrshort{rest}~\cite{fielding2000architectural},
  \textsc{CORBA}~\cite{vinoski1997corba}, etc. Dans notre travail nous
  nous limitons aux services Web (ou plus clairement
  \textit{SOAP-based web services}).

  \subsection{Services Web}
  \label{sec:ws-def}
  Les services Web sont l'approche la plus populaire pour mettre en
  œuvre une architecture \acrshort{soa}. Le terme service Web qualifie
  tout service disponible par Internet, qui utilise un format standard
  (généralement \acrshort{xml} ou \acrshort{json}) d'échange de
  messages, et qui n'est pas lié à un système d'exploitation ou un
  langage de programmation particulier.\\

  À l'origine, la technologie des services Web a été initiée par
  IBM~\cite{kreger2001web} et Microsoft, puis en partie standardisée
  par le consortium du
  \acrshort{w3c}~\footnote{\url{http://www.w3.org/}}, l'organisme
  chargé de standardiser les spécifications techniques du Web.\medskip

  D'après Curbera \emph{et al.}~\cite{curbera2001web}:\bigskip

  \textit{``Un service Web est une application réseau capable
    d'interagir par le moyen des standards et des protocoles via des
    interfaces bien spécifiées, dans lesquelles il est décrit
    utilisant un langage de description fonctionnel
    standardisé.''}\bigskip

  Selon Cubera \emph{el al.}, un service Web est vu principalement
  comme une application réseau accessible à partir d'autres
  applications. Cette définition est très ouverte dans le sens ou elle
  permet de considérer toute application identifiée par une
  \acrshort{url} comme un service Web, ce qui est pas le cas pour les
  scripts \acrshort{cgi} ou les applications \textsc{CORBA} par
  exemple.\medskip

  \acrshort{w3c} fournit une définition plus spécifique aux services
  Web~\cite{WSA}:\bigskip

  \textit{``Un service Web est un système conçu pour permettre
    l'interopérabilité des applications à travers un réseau.  Il est
    caractérisé par un format de description
    interprétable/compréhensible automatiquement par la machine,
    D'autres systèmes peuvent intéragir avec le Service Web selon la
    manière prescrite dans sa description et en utilisant des messages
    SOAP, généralement transmis via le protocole HTTP et sérialisés en
    XML ou en d'autres standards du Web.''} \bigskip

  Cette définition souligne les caractéristiques suivantes des
  services Web~\cite{fremantle2002enterprise}:

  \renewcommand{\descriptionlabel}[1]{\hspace{0.5cm}\textbullet~\textsf{#1}}
  \begin{description}
  \item[Basés sur des protocoles Internet]: L'utilisation de
    \acrshort{http} pour le transport des informations permet de
    traverser les contrôles d'accès dans un environnement hétérogène.

  \item[Interopérables]: Le standard
    \acrshort{soap}~\cite{box2000simple} définit comme étant un
    protocole destiné à l'échange de messages structurés véhiculés
    généralement sur \acrshort{http} et sérialisé en \acrshort{w3c},
    permettant ainsi le support pour l'interopérabilité.

  \item[Basés sur XML]: Le métalangage de balises \acrshort{w3c}
    \textit{eXtensible Markup Language} est un standard Web ouvert par
    \textsc{W3C}~\cite{bray1998extensible} qui offre un cadre standard
    pour la définition de documents interprétables par des machines.
  \end{description}
  \enddescription

  M. P. Papazoglou~\cite{papazoglou2003service} apporte une autre
  définition détaillée de services Web (traduction par
  \cite{driss2011approche}):\bigskip

  \textit{``Les services Web sont des éléments auto-descriptifs et
    indépendants des plates-formes permettent la composition faible
    coût d’applications distribuées. Les services Web effectuent des
    fonctions allant de simples requêtes aux processus métiers
    complexes. Les services Web permettent aux organisations d'exposer
    leurs programmes résultats sur Internet (ou sur un intranet) en
    utilisant des langages (basés sur XML) et des protocoles
    standardisés et de les mettre en œuvre via une interface
    auto-descriptive basée sur des formats standardisés et
    ouverts.''}\bigskip

  L'approche de service Web vise essentiellement quatre objectifs
  fondamentaux expliquant son grand succès \cite{driss2011approche}:

  \renewcommand{\descriptionlabel}[1]{\hspace{0.5cm}\textbullet~\textsf{#1}}
  \begin{description}
  \item [L'interopérabilité:] l'interopérabilité permet à des
    applications écrites dans des langages de programmation différents
    et s'exécutant sur des plates-formes différentes pour communiquer
    entre elles.

  \item [le couplage faible:] Le couplage est une métrique indiquant
    le niveau d'interaction entre deux ou plusieurs composants
    logiciels. Deux composants sont dits couplés s'ils échangent de
    l'information.

  \item [La réutilisation:] L'avantage de la réutilisation est qu'elle
    permet de réduire le coût de développement en réutilisant
    des composants déjà existants.

  \item [La découverte et la composition automatique:] La découverte
    et la composition sont des étapes importantes qui permettent la
    réutilisation des services. En effet il faudra être en mesure de
    trouver et de composer un service afin de pouvoir en faire
    usage. En exploitant les technologies offertes par
    \textit{Internet} et en utilisant un ensemble de standards pour la
    publication, la recherche et la composition.
  \end{description}
  \enddescription

  Techniquement, un service Web est donc une entité logicielle offrant
  une ou plusieurs fonctionnalités allant des plus simples aux plus
  complexes. Ces entités sont \emph{décrites}, \emph{publiées},
  \emph{découvertes}, \emph{invoquées} et \emph{composées} grâce à
  l'utilisation de Web (avec \acrshort{http}) comme infrastructure de
  communication ainsi que de formats de données standardisés basés sur
  \acrshort{xml}.

\section{Standards de services Web}
\label{sec:ws-standards}
Les services Web sont construits autour de standards qui sont
\acrshort{soap}, \acrshort{wsdl} et \acrshort{uddi}, assurant
respectivement leur \emph{communication}, leur \emph{description} et
leur \emph{découverte}. La figure~\ref{fig:ws-standards-relationships}
illustre les relations sous-jacentes entre ces spécifications et la
contribution de chacune d'entre elles dans le processus de mise en oeuvre
d'une architecture à base de services Web.\bigskip

%!TEX root = ../main.tex
\begin{figure}[h]
    \centering
    \includegraphics[width=0.8\textwidth]{figs/ws-standards-relationships.eps}
    \caption{Les relations entre les standards de services
      Web.~\cite{erl2004service}}
    \label{fig:ws-standards-relationships}
\end{figure}

%%% Local Variables:
%%% mode: latex
%%% TeX-master: "../main"
%%% End:


Dans la suite de cette section, nous présentons brièvement ces trois
spécifications, leurs origines, structures et fonctionnalités de base.

\newpage
  \subsection{Communication: SOAP}
  \label{sec:soap}
  Développé par IBM et Microsoft \cite{box2000simple}, le langage
  \acrshort{soap} est une recommandation \acrshort{w3c}
  \cite{mitra2003soap} qui le définit comme étant un protocole destiné
  à l'échange de messages structurés, permettant d'invoquer des
  applications sur des réseaux distribués. \acrshort{soap} est basé
  sur \acrshort{xml} pour mettre en place un mécanisme valable
  d'échange de données indépendant du modèle de programmation de
  l'application et du système d'exploitation.\medskip

  Un message \acrshort{soap} est un document \acrshort{xml} constitué
  d'une enveloppe \acrshort{soap} obligatoire, d'une en-tête
  \acrshort{soap} facultative et d'un corps \acrshort{soap} obligatoire
  (voir la figure \ref{fig:soap-message-structure}):\bigskip

  \input{figs/soap-message-structure}\bigskip

  % TODO: review
  \renewcommand{\descriptionlabel}[1]{\hspace{0.5cm}\textbullet~\textsf{#1}}
  \begin{description}
  \item[Enveloppe \texttt{<Enveloppe>}]: L'élément racine du message
    \acrshort{soap}, définissant le contexte du message, son
    destinataire et son contenu, il englobe l'en-tête et le corps.

  \item[En-tête \texttt{<Header>}]: Un mécanisme générique permettant
    d'ajouter des fonctions à un message \acrshort{soap} d'une manière
    modulaire sans accord préalable entre les parties en
    communication. Des exemples d'extension qui peuvent être
    implémentés comme des en-têtes peuvent être des authentifications, des
    transactions, des paiements.

  \item[Corps \texttt{<Body>}]: Contient les informations obligatoires
    destinées à l'ultime destinataire du message, il sert comme un
    conteneur pour les informations mandataires à l'intention du
    récepteur du message. \acrshort{soap} définit un élément pour le
    corps, qui est l'élément \texttt{<Fault>} utilisé pour rapporter
    les erreurs.
  \end{description}
  \enddescription

  \subsection{Description: WSDL}
  \label{sec:wsdl}
  Le langage de description de services Web \acrshort{wsdl}
  \cite{christensen2001web, chinnici2007web} est une recommandation du
  \acrshort{w3c}, maintenant dans sa deuxième version.
  \acrshort{wsdl} est basé sur \acrshort{xml} pour décrire les
  fonctions opérationnelles du service Web. La description
  \acrshort{wsdl} est composée d'une interface et des
  implémentations. L'interface est une définition abstraite et
  réutilisable du service qui peut être référencée par plusieurs
  implémentations.\medskip

  \acrshort{wsdl} fournit un modèle ainsi qu'un langage basé sur
  \acrshort{xml} de description de services Web. Un fichier
  \acrshort{wsdl} comprend une description des fonctionnalités d'un
  service, mais il ne se préoccupe pas de l'implantation de celles-ci.
  Il contient aussi des informations concernant la localisation du
  service, ainsi que les données et les protocoles à utiliser pour
  l'invoquer. En pratique, le document \acrshort{wsdl}
  \footnote{\url{http://www.w3.org/TR/wsdl20/}} est un document
  \acrshort{xml} qui se divise en deux parties
  \cite{elie2010}:\medskip

  \input{figs/wsdl-document-structure.tex}\medskip

  \SpecialItem
  \begin{itemize}
  \item La définition \textbf{abstraite} de l'interface du service
    avec les opérations supportées par le service Web, ainsi que leurs
    paramètres et les types de données.

  \item La définition \textbf{concrète} de l'accès au service avec la
    localisation par une adresse réseau du fournisseur de service et
    les protocoles spécifiques d'accès.\medskip
  \end{itemize}
  \enddescription

  La partie abstraite d'un document \acrshort{wsdl} contient deux
  sous-parties:\medskip

  \renewcommand{\descriptionlabel}[1]{\hspace{0.5cm}\textbullet~\texttt{#1}}
  \begin{description}
  \item[<Types>]: décrit sous la forme d'un schéma \acrshort{xml} les
    types de données échangées entre le client et le fournisseur de
    service.

  \item[<Interface>]: Les interfaces \acrshort{wsdl} offrent une
    manière abstraite pour décrire la fonctionnalité du service, ils
    définissent les opérations (éléments \texttt{<operation>}) en terme de
    paramètres d'entrée et de sortie sous forme d'un modèle
    d'échange de message \textit{(message exchange
      pattern)}. \acrshort{wsdl} contient huit modèles de messages
    prédéfinis, mais on peut facilement définir de nouveaux.\medskip
  \end{description}
  \enddescription

  La définition concrète d'un document \acrshort{wsdl} est constituée
  de:\medskip

  \renewcommand{\descriptionlabel}[1]{\hspace{0.5cm}\textbullet~\texttt{#1}}
  \begin{description}
  \item[<Binding>]: L'élément \texttt{Binding} reprend les opérations
    de l'élément \texttt{<Interface>} et leurs associe un protocole de
    transfert et des spécifications des formats de données de message.

  \item[<Service>]: Cet élément définit la localisation du service Web
    décrit. Pour chaque interface décrite, un élément service lui est
    associée. Le sous-élément \texttt{<endpoint>} définit un port
    d'accès en référençant l'élément \texttt{<binding>} associé et en
    déclarant l'\acrshort{url} localisant le service (avec l'attribut
    \texttt{<address>}).
  \end{description}
  % TODO: parler sur la limitation de la description syntaxique de WSDL,
  % réferencer les \ref{sec:ws-description}

  \newpage
  \subsection{Découverte: UDDI}
  \label{sec:uddi}
  \acrshort{uddi} \cite{clement2004uddi} est une standardisation pour
  la publication et la découverte de services Web, initialement conçue
  et spécifiée par le Consortium de standards
  OASIS\footnote{\url{https://www.oasis-open.org}}, il est le résultat
  d'un accord d'un ensemble d'industriels
  Ariba\footnote{\url{http://www.ariba.com/}}, IBM, Microsoft, etc en
  vue de devenir le registre standard de la technologie de services
  Web.\medskip

  \acrshort{uddi} complète les technologies basiques de services Web
  par la création d'un \textit{annuaire} permettant de localiser
  les services web recherchés sur le réseau. Les services référencés
  dans \acrshort{uddi} sont accessibles par l'intermédiaire du
  protocole de communication \acrshort{soap}, et la publication des
  informations concernant les fournisseurs et les services doit être
  spécifiée en \acrshort{xml} afin que la recherche et l'utilisation
  soient faites de manière \textit{dynamique} et
  \textit{automatique}.\medskip

  Un registre \acrshort{uddi} peut appartenir à un domaine public
  comme \textit{Internet} ou tout autre réseau accessible à un nombre
  non limité d'utilisateurs, comme il peut appartenir à un domaine
  restreint comme l'\textit{Intranet} d'une entreprise ou d'un groupe
  d'entreprises. Les données stockées dans l'\acrshort{uddi} sont
  structurées (en \acrshort{xml}) et organisées en trois parties
  connues:\medskip

  \renewcommand{\descriptionlabel}[1]{\hspace{0.5cm}\textbullet~\textsf{#1}}
  \begin{description}
    \item[Pages blanches]: fournissent des descriptions générales sur
      les fournisseurs de services à savoir le nom de l'entreprise qui
      fournit le service, son identificateur commercial, ses adresses,
      etc.

    \item[Pages jaunes]: comportent des descriptions détaillées sur
      les fournisseurs de services catalogués dans les pages blanches
      d'une façon taxonomique (selon secteurs d'activités par
      exemple).

    \item[Pages vertes]: fournissent des informations techniques sur
      les services Web catalogués. Ces informations incluent la
      description du service, les adresses \textsc{URL}, du processus
      de son utilisation et des protocoles utilisés pour son
      invocation.
  \end{description}

\section{Description sémantique de services Web}
\label{sec:ws-description}
La description de services Web est une étape primordiale dans
le cycle de vie d'une application à base de services. Elle permet
d'une manière claire et structurée au fournisseur de services
de communiquer au client les spécifications pour invoquer un service
Web. Le \acrshort{w3c} propose de décrire syntaxiquement les services
Web en utilisant le standard \acrshort{wsdl} \cite{christensen2001web,
  chinnici2007web} et de les publier dans l'annuaire \acrshort{uddi}
\cite{clement2004uddi}. Malgré les améliorations apportées au standard
\acrshort{wsdl} dans sa deuxième version \cite{chinnici2007web}, la
description des services reste insuffisante lors des processus de
sélection, découverte, et composition. Pour pallier cette
difficulté, plusieurs approches \cite{sivashanmugam2003adding,
  mcilraith2001semantic, mcilraith2003bringing, fensel2002web,
  paolucci2002semantic} proposent de rajouter une couche sémantique au
dessus de \acrshort{wsdl} complétant la description syntaxique par des
précisions sémantiques. Les services Web enrichis par des métadonnées
supplémentaires exprimant leur sémantique sont appelés \textit{les
services Web sémantiques}.\medskip

\input{figs/www-to-sws.tex}

Les services Web sémantiques sont le résultat de la convergence de la
technologie de services Web et du Web sémantique (illustré dans La
figure\ref{fig:www-to-sws}), cette combinaison des technologies
va permettre de:

\begin{itemize}\renewcommand\labelitemi{--}
\item Automatiser l'invocation et l'exécution du service Web par un
  service client ou par un agent.

\item Automatiser la découverte dee services.

\item La composition automatique des services Web.
\end{itemize}
\enddescription

Dans cette section, nous allons présenter les diverses approches
sémantiques visant à préciser la description d'un service en insistant
sur les approches d'annotation sémantique
(\ref{sec:semantic-annotation}) et sur les ontologies de services
(\ref{sec:ontologies-services}). La notion du Web sémantique est
abordée brièvement dans l'annexe \ref{annexe:semantic-web}.

  \subsection{Annotations sémantiques}
  \label{sec:semantic-annotation}
  L'annotation sémantique consiste à enrichir et à compléter la
  description d'un service. Elle établit des correspondances entre des
  éléments de la description et des concepts d'un ensemble
  d'ontologies de références. Une ontologie de référence permet de
  représenter un domaine par des structures interprétables par une
  machine. Deux modèles principaux suivent l'approche d'annotation
  sémantique, à savoir \textsc{WSDL-S} et \acrshort{sawsdl}
  \cite{elie2010}.

    \subsubsection{WSDL-S}
    \textsc{WSDL-S} \cite{akkiraju2005web} est le résultat d'un
    travail collaboratif entre IBM, le laboratoire LSDSI et
    l'université de Geogia. La spécification est devenue une
    recommandation \acrshort{w3c} depuis 2005
    \footnote{\url{http://www.w3.org/Submission/WSDL-S/}}. Son
    objectif principal est de fournir un processus d'annotation
    sémantique compatible avec les technologies
    existantes. Pratiquement, le méta-modèle \textsc{WSDL-S} repose
    sur les éléments extensibles du modèle \acrshort{wsdl} introduits
    dans sa deuxième version en rajoutant trois éléments majeurs
    \texttt{<category>}, \texttt{<precondition>} et \texttt{<effect>},
    et deux attributs, \texttt{modelReference} et
    \texttt{schemaMapping}.\medskip

    \renewcommand{\descriptionlabel}[1]{\hspace{0.5cm}\textbullet~\texttt{#1}}
    \begin{description}
    \item [<category>:] est un sous-élément de \texttt{<portType>}. Il
      précise la catégorie d'un service lors de la publication dans un
      annuaire ou registre.

    \item [<precondition>:] sous-élément de \texttt{<operation>},
      indique les préconditions à vérifier pour que l'opération
      s'exécute comme prévu.

    \item [<effects>:] sous-élément de \texttt{<operation>}, indique
      les effets de l'exécution de l'opération.

    \item [modelreference:] Un attribut qui peut être ajouté à élément
      dans une grammaire \acrshort{xml} et aux éléments
      \texttt{<category>}, \texttt{<precondition>} et
      \texttt{<effects>}. Il indique une association avec un élément
      correspondant dans une ontologie donnée.

    \item [schemamapping:] Un attribut peut être ajouté à un élément
      dans une grammaire \acrshort{xml}. Il permet de décrire les
      correspondances entre la grammaire annotée et les ontologies de
      référence.
    \end{description}
    \enddescription
    \bigskip

    Les éléments permettent de rajouter des informations
    qui n'étaient pas prises en compte dans \acrshort{wsdl} comme
    \emph{les préconditions} et \emph{les effets} d'une opération,
    tandis que les attributs permettent de référencer des concepts
    dans des ontologies de référence, ces préconditions et effets.

    \subsubsection{SAWSDL}
    \acrshort{sawsdl} \cite{kopecky2007sawsdl} est un langage
    d'annotation sémantique de description de services Web qui hérite
    de \textsc{WSDL-S}. Issu d'une initiative du groupe de travail
    d'annotations sémantiques pour \acrshort{wsdl}
    \footnote{\url{http://www.w3.org/TR/sawsdl/}} et soumis au
    \acrshort{w3c} en 2007, \acrshort{sawsdl} définit un mécanisme
    pour annoter sémantiquement les interfaces et les opérations
    \acrshort{wsdl}, ainsi que les types \textit{``XML schema''} en
    les reliant à des concepts dans une ontologie.\medskip

    Cette annotation repose sur la définition d'attributs étendant le
    standard de description en référençant des ontologies
    préexistantes. Le mécanisme d'annotation de \acrshort{sawsdl} est
    indépendant de tout langage de représentation
    \cite{lopez2008selection} d'ontologies.\medskip

    \acrshort{sawsdl} propose trois sortes d'annotations sémantiques
    pour référencer des concepts de modèles définis à l'extérieur du
    document \acrshort{wsdl}, une pour identifier le concept
    sémantique (représentée par l'attribut \texttt{modelReference}) et
    deux autres (\texttt{liftingSchemaMapping} et
    \texttt{loweringSchemaMapping}) pour spécifier la correspondance
    (\emph{Mapping}) entre les données sémantiques et les éléments
    \acrshort{xml}:\bigskip

    \renewcommand{\descriptionlabel}[1]{\hspace{0.5cm}\textbullet~\texttt{#1}}
    \begin{description}
    \item [modelReference:] Il permet d'associer un composant
      \acrshort{wsdl} ou une instance\acrshort{xml} schema à un concept
      d'une ontologie.

    \item [liftingSchemaMapping:] Il définit comment un élément
      \acrshort{xml} peut être transformé en une donnée conforme
      à un certain modèle sémantique.

    \item [loweringSchemaMapping:] Dans le sens inverse que le
      précédent, cet attribut définit comment les données dans un
      modèle sémantique sont transformées en instances \acrshort{xml}.
    \end{description}
    \enddescription

  \subsection{Ontologies de services}
  \label{sec:ontologies-services}
  Une ontologie de services saisit les différents aspects liés à la
  description de services et leurs utilisations à travers un ensemble de
  concepts, de propriétés et de relations entre eux
  \cite{elie2010}. Deux modèles d'ontologies de services sont décrits
  dans cette sous-section \textsc{OWL-S} et \acrshort{wsmo}.

    \subsubsection{WSMO}
    \label{sec:wsmo}
    \acrshort{wsmo} \cite{de2005web} est un modèle conceptuel basé
    sur la spécification \acrshort{wsmf} \cite{fensel2002web} (voir
    \ref{sec:wsmf}) pour la description des divers aspects liés aux
    services Web sémantiques. Le but de \acrshort{wsmo} est
    d'automatiser le cycle de vie de services web (publication,
    sélection, découverte, composition, etc.) afin de résoudre le
    problème de l'intégration de services Web en définissant une
    technologie cohérente pour description de services Web
    sémantique.\medskip

    Pour formaliser \acrshort{wsmo}, le groupe de travail a mis au point
    le langage de modélisation \acrshort{wsml} \cite{de2006web} et a
    défini plusieurs variantes de celui-ci, chacune basée sur différents
    formalismes.\medskip

    \acrshort{wsmo} est constitué de quatre composants: les services
    web, les buts, les ontologies et les médiateurs. Ce modèle permet
    de réaliser un couplage faible entre les services web en utilisant
    un ensemble de médiateurs. Ces derniers assurent les tâches
    d'intégration d'ontologies, de découverte de services, de
    composition, etc.

    \newpage
    \subsubsection{OWL-S}
    \label{sec:owl-s-1}
    \textsc{OWL-S} \cite{martin2004owl} désigné par \textsc{DAML-S}
    \cite{ankolekar2002daml} dans les versions antérieures, est un
    langage issue des travaux de \acrshort{darba}
    \footnote{\url{http://www.darpa.mil/}} et son programme
    \acrshort{daml} \footnote{\url{http://www.daml.org/services/}} en
    collaboration avec des chercheurs de plusieurs universités et
    organisations. Il a été intégré au consortium \acrshort{w3c} en
    2004 au sein du groupe d'intérêt sur les services Web sémantiques,
    lors de la recommandation du langage \textsc{OWL}
    \cite{horrocks2002daml+oil, mcguinness2004owl}. Ankolekar \emph{et
      al.}  \cite{ankolekar2002daml} présentent une ontologie pour les
    services web dans le but d'automatiser la \emph{découverte},
    \emph{l'invocation}, la \emph{composition} et la
    \emph{surveillance} de l'exécution des services
    \cite{mcilraith2003bringing}, les auteurs reprennent la notion de
    classes d'\textsc{OWL} et proposent l'ontologie
    \textsc{OWL-S}.\bigskip

    \input{figs/owls.tex}

    L'objectif principal de ces recherches est d'établir une
    plate-forme dans laquelle les descriptions de services Web sont
    partagées en utilisant une ontologie standard, constituée d'un
    ensembles de classes de base et des propriétés pour résoudre les
    ambiguïtés et de rendre la description d'un service compréhensible
    par la machine.\bigskip

    la figure \ref{fig:owl-s} décrit la structure tripartie d'une
    ontologie \textsc{OWL-S}. Elle est composée de trois
    sous-ontologies: un \emph{service profil}, un \emph{service
      model} et un \emph{service grounding}:

    \renewcommand{\descriptionlabel}[1]{\hspace{0.5cm}\textbullet~\texttt{#1}}
    \begin{description}
    \item[ServiceProfile]: il offre une description informelle des
      fonctionnalités rendues par le service (\verb|serviceName|,
      \verb|textDescription|) et des informations concernant son
      fournisseur (\verb|contactInformation|). D'un autre côté, il
      spécifie des fonctionnalités offertes par le service (
      comportement fonctionnel) en terme de transformation
      d'informations dénotées par les \textit{entrées/sorties}
      \textsc{(I/O)} (\verb|hasInput|, \verb|hasOutput|) et de
      changement d'état après l'exécution du service dénoté par les
      \textit{préconditions/effets} \textsc{(P/E)}
      (\verb|hasPrecondition|, \verb|hasResult|).\medskip

      Du point de vue découverte et composition, Le
      \verb|ServiceProfile| est la partie la plus importante de la
      définition du service \textsc{OWL-S}.

    \item[ServiceModel]: il décrit le fonctionnement du service en
      indiquant comment les résultats sont produits étape par étape
      précisant la façon qu'un client peut interagir avec le service afin
      d'atteindre sa fonctionnalité. Ceci est fait en exprimant la
      transformation de données avec \textit{Entrées/Sorties}
      \textsc{(I/O)} et la transformation de l'état avec
      \textit{pré-conditions/effets} \textsc{(P/E)}.\medskip

      le \verb|ServiceProfile| est généralement considéré comme un
      sous-ensemble du \verb|ServiceModel|, contenant uniquement
      l'information nécessaire pour annoncer le service Web pour une
      découverte ultérieure.

    \item[ServiceGrounding]: Permet de spécifier les détails d'accès
      au service en précisant le protocole, le format des messages, la
      sérialisation et l'adressage. Il représente une correspondance
      \textit{(mapping)} entre la définition abstraite d'un processus
      \textsc{OWL-S} décrivant le service et la définition
      \textsc{WSDL} concrète des éléments nécessaires pour interagir
      avec le service.

      % Le rôle de mise en correspondance est de principalement
      % combler l'écart entre la description sémantique du service Web
      % (détaillée dans lee deux premières sous-ontologies) et les modèles
      % de description de services existants qui est principalement
      % syntaxique (\acrshort{wsdl}).
    \end{description}
    \enddescription

    Le \verb|ServiceProfile| fournit éventuellement d'autres
    informations supplémentaires sur le service comme la qualité qu'il
    assure en terme de temps de réponse et de coût, une classification
    possible d'un service (\verb|serviceCategory|), et un paramètre
    générique \verb|serviceParameter|.

\newpage
\section{Découverte de services web}
\label{sec:ws-discovery}
La découverte de services Web présente un axe de recherche très
important. Divers mécanismes de découverte ont été proposés dans la
littérature et plusieurs définitions sont attribuées à ce
concept. Booth \textit{el al.} \cite{booth2004web} décrivent le
processus de découverte comme étant l'acte de \textit{``localisation
  d'une description compréhensible par la machine d'un service
  éventuellement inconnu au préalable écrivant certains critères
  fonctionnels.''}

\input{figs/ws-discovery-with-matching.tex}

Essentiellement, un processus de découverte de service web s'effectue
en deux phases principales (illustrés dans la figure
\ref{fig:ws-discovery-with-matching}). Premièrement, un module de mise
en correspondance (\textit{Matching}) renvoie tous les services Web
candidats selon le degré de similitude entre les services disponibles
et les exigences spécifiées par la requête (aspect fonctionnel). La
deuxième étape consiste à sélectionner et invoquer le service le plus
satisfaisant (aspect \textit{non-fonctionnel}).

Plusieurs critères peuvent être utilisés pour catégoriser les
approches de découverte des services \cite{ mohebbi2010comparative,
  elie2010, bitar2014cbr4wsd}):

  \begin{enumerate}
  \item la localisation de services
    (\textit{centralisation/décentralisation} des annuaires).
  \item Le degré d'automatisation du processus de la découverte.
  \item le principe de l'algorithme de \textit{Matching}.
  \end{enumerate}

  Cette section vise à présenter les différentes approches de
  découverte des services Web proposées dans la littérature. Ces
  modèles ont été classés selon les critères déjà mentionnés, à
  savoir la localisation \ref{sec:ws-localisation} (centralisée,
  distribuée), le degré d'automatisation du processus
  \ref{sec:ws-desc:manual-vs-auto} (manuel, automatique), et le niveau
  de \textit{Matching} \ref{sec:ws-matching} (syntaxique ou
  sémantique).

\subsection{Localisation de services}
\label{sec:ws-localisation}
La découverte consiste principalement à localiser les descriptions de
services répondant à une requête cliente. Les approches de
localisation de services récurrentes dans la littérature sont classées
en deux catégories, à savoir les approches centralisées et les
approches décentralisées ou distribuées \cite{garofalakis2004web}.

    \subsubsection{Approches centralisées}
    \label{sec:ws-localisation-centr}
    Les premières versions d'\acrshort{uddi} \cite{clement2004uddi}
    (présenté dans \ref{sec:uddi}) reposent sur une approche
    \textit{centralisée} de publication et découverte de services
    Web. le registre \acrshort{uddi} définit un modèle de
    représentation des données et des métadonnées nécessaires à la
    publication, il s'appuie sur \textit{un seul annuaire} qui peut être
    géré par un module de mise en correspondances
    \textit{(matchmaker)}.

    \subsubsection{Approches décentralisées}
    \label{sec:ws-localisation-distr}
    Les approches décentralisées de découverte de services
    \cite{rompothong2003query, sivashanmugam2004discovery,
      paolucci2003using, schmidt2004peer, verma2005meteor,
      sahin2005spider} consistent à mettre en place une
    \textit{fédération} d'annuaires \acrshort{uddi} agissant comme une
    couche d'abstraction reliant plusieurs instances d'annuaires. La
    plupart des approches distribuées sont basées sur des systèmes
    pair-à-pair (\textit{peer to peer}) incluant
    \cite{schmidt2004peer, verma2005meteor, sahin2005spider}.

    La dernière version d'\acrshort{uddi}
    \cite{oasis2005specification} (\textit{3.0.2}) reprend le principe
    d'une fédération d'annuaires et décrit un annuaire \acrshort{uddi}
    comme un ensemble de nœuds \textit{(UDDI nodes)} tel que chaque
    nœud fait partie d'un seul annuaire et possède une copie répliquée
    de schéma global de la fédération. Les nœuds d'un annuaire
    collaborent pour gérer un ensemble de structures de données
    \acrshort{uddi} pour permettre une meilleure gestion des requêtes.

  \subsection{Découverte manuelle/automatique}
  \label{sec:ws-desc:manual-vs-auto}
  le degré d'intervention de l'utilisateur dans la découverte varie
  selon la richesse sémantique de la description du service. Nous
  pouvons distinguer les approches manuelles et semi-automatiques d'une
  part, et les approches automatiques d'une autre part
  \cite{elie2010,garofalakis2004web}.

    \subsubsection{Approches manuelles et semi-automatiques}
    Dans une découverte manuelle, le client utilise un service de
    découverte pour localiser et sélectionner manuellement une
    description de service qui répond à certains critères
    fonctionnels. La recherche de descriptions dans cette approche est
    souvent basée sur une simple comparaison syntaxique entre les mots
    clés de la requête et les descriptions de services disponibles, ce
    qui nécessite l'intervention de l'utilisateur pour vérifier la
    pertinence et la fiabilité des résultats de la recherche et
    sélectionner le service Web qui répond au mieux à ses exigences.

    \subsubsection{Approches automatiques}
    La découverte automatique de services qui répond à un besoin donné
    est considéré comme une étape cruciale vers l'intégration
    dynamique et évolutive de services Web. On entend par découverte
    automatique la possibilité de localiser automatiquement un service
    Web qui répond à des besoins particuliers. Différentes
    approches ont été proposées pour réaliser la
    découverte dynamique de services ( \cite{paolucci2002semantic,
      bernstein2002discovering, mandell2003bottom,
      benatallah2005automating,keller2005automatic}).\medskip

    Dans \cite{mandell2003bottom} les auteurs présentent une approche
    d'automatisation de la découverte de services modélisés en
    \acrshort{bpel} (voir \ref{sec:bpel}) dans le but d'une
    composition plus dynamique. Les auteurs proposent d'étendre la
    description \textsc{BPEL} par l'intégration d'une description
    sémantique \textsc{OWL-S} (voir \ref{sec:owl-s}) de type
    \textit{service profile} et la mise en place d'un module de
    \textit{Matching} sémantique équipé par un raisonneur
    automatique. Autrement, Keller \textit{et al}
    \cite{keller2005automatic} étudient un modèle de localisation
    automatique de services Web décrits via le modèle \acrshort{wsmo}.

  \subsection{Matching de services Web}
  \label{sec:ws-matching}
  Le \textit{Matching} (ou le \textit{Matchmaking}) de services Web
  est défini comme un processus qui nécessite un annuaire de services
  pour prendre une requête en entrée, et de retourner les services les
  plus pertinents qui peuvent satisfaire les exigences spécifiées dans
  cette requête \cite{li2004software}. Cette opération nécessite la
  recherche de similarités indiquant le degré de rapprochement entre
  les paramètres descriptifs \textit{fonctionnels} (les paramètres
  \textit{entrées/sorties}) et \textit{non fonctionnels} (décrivant la
  qualité de service, coût, etc.) du service requis et ceux des
  services offrets.

  \input{figs/matching-general.tex}

  Dans le contexte d'un processus de découverte, on parle souvent
  d'un \textit{Matching} \textbf{vertical} des services (contrairement
  au \textit{Matching} horizontal lors d'une opération de composition
  des services Web), La figure \ref{fig:matching-general} illustre une
  opération de \textit{Matching} vertical entre deux services.\medskip

  Le degré de \textit{Matching} constitue un critère primaire dans le
  processus de découverte des service puisque l'efficacité du
  processus de découverte dépend essentiellement de la manière dont
  le \textit{Matching} est effectué. Le calcul de similarité peut être
  basé sur des données syntaxiques ou sémantiques plus expressives
  \cite{elie2010}.

    \subsubsection{Matching syntaxique}
    \label{sec:matching-syntaxique}
    Les approches de \textit{Matching} syntaxique sont basées sur la
    description syntaxique de services (\acrshort{wsdl}) référencée
    dans un annuaire \acrshort{uddi} \cite{clement2004uddi}. Le client
    envoie une requête constituée de mots clés, cette requête est
    ensuite comparée avec les mots clés du registre
    \acrshort{uddi}. Un ensemble de descriptions de services Web sous
    forme de documents \acrshort{wsdl} (ou un ensemble des
    \textsc{URL} de ces derniers) est ensuite retourné comme résultat de
    recherche, le client sélectionne le service Web qui répond au
    mieux à ses exigences.\medskip

    Le système décrit par Liang \textit{et al.}
    \cite{DBLP:journals/jwsr/LiangCSCL04} est un exemple d'une
    découverte syntaxique des services Web. Ce système permet une
    découverte et une composition semi-automatique par l'utilisation
    de la base de données lexicale \textit{WordNet}
    \footnote{\url{http://wordnet.princeton.edu/}}
    \cite{miller1990introduction} pour enrichir les mots clés de la
    requête avec des synonymes.\medskip

    Sajjanhar \textit{et al.} \cite{sajjanhar2004algorithm} utilisent
    des méthodes d'Algèbre linéaire pour rendre le \textit{Matching}
    par mots clés plus efficace. leur approche consiste à construire
    une matrice contenant des informations sur tous les services Web
    qui ont des mots communs dans leurs descriptions. Ensuite, une
    décomposition en valeurs singulières est appliquée à cette matrice
    pour obtenir toute description de service Web qui a une relation
    avec la requête de recherche.\medskip

    Malgré sa simplicité et sa facilité d'implémentation, les
    approches syntaxiques présentent plusieurs limitations:

    \begin{itemize}\renewcommand\labelitemi{--}
    \item Un faible taux de précision à cause de l'ambiguïté du
      langage naturel.
    \item La nécessité de l'intervention de l'utilisateur lors de
      la sélection.
    \end{itemize}

    \subsubsection{Matching sémantique}
    \label{sec:matching-semanique}
    Des travaux récents \cite{paolucci2002semantic,
      benatallah2005automating, keller2004wsmo, benatallah2003request,
      jaeger2005ranked} se sont focalisés sur la description
    sémantique de services Web dans le but d'adresser les
    insuffisances des approches purement syntaxiques. Les ontologies
    sont les modèles les plus utilisés pour la représentation
    sémantique de services Web. Dans ce type de découverte, le
    prestataire doit fournir des annotations sémantiques pour ses
    services publiés.\medskip

    La description sémantique se rapporte à la définition abstraite de
    services comprenant les types d'\textit{entrées/sorties}. Un
    \textit{Matching} sémantique consiste à identifier un certain
    degré de similitude entre les différents concepts sémantiques qui
    décrivent le service requis (la requête) et celles de services
    publiés.\bigskip

    Paolucci \emph{el al.} \cite{paolucci2002semantic} présente une
    approche de \textit{Matching} sémantique entre des services
    définis avec \textsc{DAML-S} (\textsc{OWL-S}). Ils
    décrivent un algorithme qui assure la correspondance entre les
    profils de services publiés, décrit par des \textit{service
      profile}, et l'objectif du client défini par une structure
    \textit{service profile}.\medskip

    L'algorithme de Paolucci \emph{el al.} consiste à comparer les paramètres
    \textit{entrées/sorties} des profils publiés ($P_S$) avec ceux du
    profil requis, ou recherché ($R_S$). Le calcul de
    \textit{Matching} est basé sur la distance minimale entre les
    concepts dans la taxonomie de l'ontologie. Un service publié est
    considéré comme un service qui répond bien à une requête lorsque
    toutes les sorties de la requête correspondent à des sorties du
    service publié, et toutes les entrées du service publié
    correspondent à des entrées de la requête.\medskip

    Les auteurs dénotent quatre niveaux de \textit{Match} (dans ce cas,
    on applique \textit{Match($R_S, P_S$)}):

    \renewcommand{\descriptionlabel}[1]{\hspace{1cm}--~\textsf{#1}}
    \begin{description}
    \item [Exact:] Les paramètres de sorties du profil recherché sont
      équivalents ou constituent une sous-classe des paramètres du
      profil publié.

    \item [PlugIn:] Les paramètres de sorties du profil recherché sont
      subsumés (\textit{englobés}) par les paramètres du profil publié
      \textbf{($P_S$ subsumes $R_S$)}.

    \item [Subsumes:] Les paramètres de sorties du profil recherché
      subsument (\textit{englobent}) les paramètres du
      profil publié \textbf{($R_S$ subsumes $P_S$)}. Dans ce cas, le
      service publié répond partiellement au besoin recherché.

    \item [Fail:] Aucun lien de subsomption entre les deux.
    \end{description}
    \enddescription

    Les auteurs notent que le processus de \textit{Matching} décrit
    n'est pas symétrique, En effet \textit{Match($w_1, w_2$) <>
      Match($w_2, w_1$)}, parce que l'ordre dans lequel les paramètres
    d'entrée et de sortie sont comparés est important, dans le cas de
    \textit{Matching} des paramètres d'entrée, on applique un
    \textit{Match($P_E, R_E$)}.\bigskip

    À l'instart de \cite{paolucci2002semantic}, Jaeger \emph{el al.}
    \cite{jaeger2005ranked} proposent une approache basée sur un
    \textit{Matching} sémantique d'\textit{entrées/sorties}, Partant
    d'un appariement entre deux profils de service spécifiés en
    \textit{OWL-S}, le processus de \textit{Matching} prend en compte
    les \textit{entrées/sorties}, la catégorie de service et des
    éléments prédéfinis par l'utilisateur.\bigskip

    Benatallah \emph{el al.} \cite{benatallah2005automating,
      benatallah2003request} étendent le modèle de Paolucci \emph{el
      al.}  pour aller au delà de la simple notion de subsomption. Ils
    présentent une approche de réécriture de concepts basée sur la
    notion de meilleure couverture et sur l'opération de différence
    entre concepts; Étant donné un profil de service recherché désigné
    par une requête \textit{R} et une ontologie de domaine \textit{O},
    l'objectif est de découvrir la meilleure combinaison de services
    satisfaisant au maximum les sorties de \textit{R}. Ils désignent
    ce processus par ``la meilleure couverture de \textit{R} en
    utilisant \textit{O}'' \cite{elie2010}.\bigskip

    Keller \emph{el al.} \cite{keller2004wsmo, keller2005automatic}
    présentent une approche similaire à celle de
    \cite{paolucci2002semantic} pour la découverte de services Web,
    Leurs travaux sont basés sur le \textit{Matching} sémantique entre
    les descriptions \acrshort{wsmo} des services.


\section*{Conclusion}
\label{sec:conclusion}
\addcontentsline{toc}{section}{Conclusion} \markboth{CONCLUSION}{}

Les services Web sont l'une des technologies les plus répandues pour
la mise en œuvre d'une architecture orientée services
(\acrshort{soa}). Ils ont des composants logiciels
\textit{interopérables} qui fournissent des fonctionnalités
accessibles via des protocoles standardisés du Web. L'apport principal
des services Web est la solution au problème d'\textit{hétérogénéité}
par rapport aux plates-formes et langages de programmation et d'avoir
un partage des fonctionnalités et facilite grandement le
développement.\medskip

Dans ce chapitre nous avons abordé dans un premier temps
l'architecture orientée services, ainsi que les notions de base
relatives aux services web \ref{sec:ws-definitions}, puis nous avons
présenté trois spécifications principales qui supportent la
technologie des services web, à savoir, \acrshort{soap} \ref{sec:soap}
pour la communication, \acrshort{wsdl} \ref{sec:wsdl} qui s'occupe de
la description de services Web et \acrshort{uddi} \ref{sec:uddi} pour
la publication. Ensuite, Nous avons décrit les langages d'annotation
sémantique de descriptions de services Web \ref{sec:ws-description}
notamment l'ontologie \textit{OWL-S}. La section
\ref{sec:ws-discovery} a donné un éclairage sur les principaux
approches pour la découverte de services Web, nous pouvons constater
qu'il existe une corrélation directe entre les approches sémantiques
de description et le degré d'automatisation du processus de découverte
de services permettant une intégration dynamique et évolutive de
services Web.

%%% Local Variables:
%%% mode: latex
%%% TeX-master: "../main"
%%% End:
