\chapter{Une méthode de composition dynamique des services Web sémantiques utilisant neo4j}

\section*{Introduction}
\addcontentsline{toc}{section}{Introduction} \markboth{INTRODUCTION}{}

\section{Reformulation mathématique du problème}
\label{sec:reform-du-probl}
% \cite{lecue2006formal}
  \subsection{Définitions de base}
  \label{sec:un-formalisme-pour}
  \subsection{Hypothèses du travail}
  \label{sec:hypotese-de-travail}
  \subsection{Objectifs}
  \label{objectifs}

\section{Example d'illustration}
\label{sec:example-dill}

\section{Annotation sémantiques des services web}
\label{sec:annot-semant-des}

\section{Architecture du système}
\label{sec:larch-du-syst}

\section*{Conclusion}
\label{sec:conclusion}
\addcontentsline{toc}{section}{Conclusion} \markboth{CONCLUSION}{}


%%% Local Variables:
%%% mode: latex
%%% TeX-master: "../main"
%%% End:
