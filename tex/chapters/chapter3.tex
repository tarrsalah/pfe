\chapter{Vers les bases de données graphe}
\label{ch:graph-db}

\section*{Introduction}
\addcontentsline{toc}{section}{Introduction} \markboth{INTRODUCTION}{}

Avec le développement rapide et continu de l'\textit{Internet} et du
\textit{cloud computing}, divers types d'applications ont émergés, ce
qui augmente l'importance accrue de la technologie de base de données,
notamment dans les aspects suivants \cite{han2011survey}:

\begin{itemize}\renewcommand\labelitemi{--}
\item Haute concurrence de lecture et d'écriture avec une faible
  latence.

\item Stockage et la gestion d'une grande masse de données \emph{(Big
    Data)}.

\item Haute scalabilité (évolutivité) et disponibilité.\medskip
\end{itemize}
\enddescription

Bien que les bases de données relationnelles ont occupé une grande
position dans le domaine de stockage de données, le modèle relationnel
commence à montrer ses limites en faisant face aux plusieurs
exigences: \cite{han2011survey}: lente lecture et l'écriture, capacité
limitée, difficulté d'expansion et scalabilité, etc. Afin de faire
face aux limitations ci-dessus, une variété de nouveaux types de bases
de données ont apparus qui s'affranchissent du modèle relationnel
pour miser sur le partitionnement horizontal et le relâchement des
contraintes d'intégrité, ce modèle émergent est appelé le modèle
\acrshort{nosql}.\medskip

Dans ce chapitre, nous exposons les différentes approches et
plateformes du stockage \acrshort{nosql} les plus connues dans
l'industrie tout en mettant l'accent sur les bases de données graphe
(et \emph{Neo4j} particulièrement). Nous commençons d'abord par
présenter les quatre catégories majeures des systèmes
\acrshort{nosql}. Ensuite, nous nous focalisons sur les \acrshort{SGBD}
orientés graphes, leurs différents modèles et implémentations ainsi
que les langages de requêtes les plus adoptés pour les interroger.

\section{L'environnement NoSQL}
\label{sec:nosql}
\acrshort{nosql} désigne une nouvelle génération de technologies de
gestion des bases de données conçu pour répondre à l'augmentation du
volume des données stockées, traitées et analysées par les entreprises
et organisations ayant une large étendue d'utilisateurs tels que
Google, Amazon, Facebook et Twitter, \acrshort{nosql} comprend une
multitude de systèmes de gestion de données qui ne sont plus fondés
sur l'architecture classique des bases relationnelles. L'unité logique
n'y est plus la table, et les données ne sont en général pas
manipulées avec \textsc{SQL}.  D'après Fowler \textit{et al.}
\cite{sadalage2012nosql}, Les caractéristiques communes des bases de
données \acrshort{nosql} sont:

\begin{itemize}\renewcommand\labelitemi{--}
\item Ne sont pas fondées sur de le modèle relationnel classique.
\item Optimisées pour les environnements distribués.
\item Construites pour les applications Web modernes avec une grande
  masse de données.
\item \emph{Schemaless} (il n'existe plus d'intégrité référentielle ou
  schéma prédéfini).
\end{itemize}

  \subsection{Les catégories des bases de données NoSQL}
  \label{sec:cat-nosql}
  Il existe dans la mouvance \acrshort{nosql} une diversité importante
  d'approches que nous classons en quatre grandes catégories
  \cite{sadalage2012nosql}: dépôts \textit {clés/valeurs}, bases
  orientées documents, orientées colonnes, et bases orientées
  graphe.\bigskip

  \textsf{Dépôts clés/valeurs}: Le principe des bases
  \textit{clé/valeur} est de stocker les données sous une forme
  simple: une \emph{clé } (chaîne de caractères) associée à une
  \emph{valeur} d'une forme libre (chaîne de caractère, ou bien un
  objet sérialisé), cette clé est utilisée pour toutes les opérations
  à effectuer sur les données telles que l'insertion, la mise à jour
  et la suppression. Bien que la structure est plus simple, la vitesse
  d'interrogation est extrêmement supérieur à la base de données
  relationnelles favorisant la scalabilité (\emph{scalability}) plus
  que la cohérence, on les retrouve très souvent comme système de
  stockage de cache ou de sessions distribuées, notamment là où
  l'intégrité relationnelle des données est non significative. Les
  solutions les plus connues, telles que \emph{Redis}, \emph{Riak} et
  \emph{Voldemort} sont principalement influencées par le projet
  \emph{Dynamo} d'Amazon \cite{decandia2007dynamo}.\bigskip

  \textsf{Orientées documents}: Cette famille de base de données est
  une évolution de la base de données \textit{clé/valeur} destinée aux
  applications qui gèrent des documents (généralement du format
  \textsc{JSON} ou \textsc{XML}) où chaque clé n'est plus associée à
  une valeur sous forme de bloc binaire mais à un document dont la
  structure reste libre (\textit{scheme-less}). L'avantage est de
  pouvoir récupérer, via une seule clé, un ensemble d’informations
  structurées de manière hiérarchique, ainsi le stockage de
  volumes très importants de données pour lesquelles la modélisation
  relationnelle aurait entraînée une limitation des possibilités de
  partitionnement et de réplication. Les deux implémentations les plus
  populaires dans cette catégorie sont \emph{CouchDB} d'Apache et
  \emph{MongoDB}.\bigskip

  \textsf{Orientées colonnes}: La famille des stockages orientés
  colonnes sont une évolution du modèle \textit{clé/valeur} utilisant
  des \textit{``familles''} afin de grouper certaines lignes. Ces
  familles peuvent devenir hiérarchiques et les relations entre les
  données deviennent explicites. Une base de données orientées
  colonnes permet de plus une compression par colonne, efficace
  lorsque les données de la colonne se ressemblent. Comme solutions,
  on retrouve principalement \emph{HBase}, \emph{HyperTable}
  (implémentations Open Source du modèle \emph{BigTable}
  \cite{chang2008bigtable} publié par Google) ainsi que
  \emph{Cassandra} (projet Apache qui respecte l'architecture
  distribuée de \emph{Dynamo} \cite{decandia2007dynamo} d'Amazon et le
  modèle BigTable de Google).\bigskip

  \textsf{Orientées graphes}: Ce paradigme est le moins connu de ceux
  de la mouvance \acrshort{nosql}. Ce modèle s'appuie principalement
  sur deux concepts: d'une part l'utilisation d'un moteur de stockage
  pour les objets (qui se présentent sous la forme d'une base
  documentaire, chaque entité de cette base étant nommée
  \emph{nœud}). D'autre part, à ce modèle, vient s'ajouter un
  mécanisme permettant de décrire les relations entre les objets
  (\emph{arcs}). Principalement, ces bases de données sont nettement
  plus efficaces que leur précedente relationnelle pour traiter les
  problématiques liées aux réseaux.  En effet, lorsqu'on utilise le
  modèle relationnel, cela nécessite un grand nombre d'opérations
  complexes (souvent, des jointures trop lentes) pour obtenir des
  résultats. Dans cette catégorie on peut citer \emph{Neo4j},
  \emph{OrientDB}, \emph{Titan} et \emph{AllegroGraph} comme les
  implémentations les plus répondues de ce modèle.

  \subsection{Le théorème de CAP }
  \label{sec:cap}
  \acrshort{cap} \cite{brewer2000towards} est l'acronyme de
  \textit{``\textbf{C}onsistency, \textbf{A}vailability and
    \textbf{P}artition Tolerance''}, ou \textit{``Cohérence,
    Disponibilité et Résistance au partitionnement''}. Ce théorème
  explique qu'il est impossible qu'un système distribué satisfasse
  simultanément aux trois contraintes suivantes:\medskip

  \renewcommand{\descriptionlabel}[1]{\hspace{1cm}\textbullet~\textsf{#1}}
  \begin{description}
  \item [Cohérence]: tous les nœuds du système voient exactement les
    mêmes données au même moment.

  \item [Disponibilité]: la perte de nœuds n'empêche pas les
    survivants de continuer à fonctionner correctement.

  \item [Résistance au partitionnement]: aucune panne moins importante
    qu'une coupure totale du réseau ne doit empêcher le système de
    répondre correctement.
  \end{description}
  \enddescription
  \bigskip

  \input{figs/cap.tex}

  \newpage

  Les bases de données relationnelles implémentent les propriétés de
  cohérence et de disponibilité (système \emph{CA}). Les bases de
  données \emph{NoSQL} sont généralement des systèmes \emph{CP}
  (cohérent et résistant au partitionnement) ou \emph{AP} (disponible
  et résistant au partitionnement), la figure \ref{fig:cap} montre
  le positionnement de quelque systèmes \emph{NoSQL} par rapport au
  théorème \acrshort{cap}.

\section{Bases de données orientées graphes}
\label{sec:graph-database-overview}
En raison du volume exponentiellement croissant de données ouvertes
disponibles, l'intérêt pour les systèmes de gestion de bases de
données graphe est grandissant. Ces systèmes permettent de manipuler
des données structurées sous forme de graphe, auxquelles aucun schéma
n'est a priori associé. Ainsi, les systèmes de gestion de bases de
données graphe permettent entre autres la gestion de données des
réseaux sociaux, de données \textit{bio-informatiques}, et de données
issues du Web sémantique tels que des documents \acrshort{rdf}
éventuellement issus de l'exploitation des données ouvertes. De
nombreux systèmes permettant la gestion des bases de données graphe
sont apparus. Il s'agit par exemple de \textit{AllegroGraph}
\textit{InfiniteGraph}, \textit{Neo4j} ou \textit{Sparksee}.\medskip

Cette section a pour but d'introduire les modèles de données utilisés
pour la persistance des données hautement connectées
\ref{sec:persistence}, ainsi que les techniques du stockage
sous-jacent et d'indexation utilisées par les différents systèmes de
gestion des bases de données graphe.
\ref{sec:graph-internals}.\medskip

  \subsection{Définitions et caractéristiques}
  \label{sec:graphdb-defs}
  % Les bases de données orientées graphes sont conçues pour modéliser
  % des réseaux de données fortement connectées et y naviguer facilement
  % en bénéficiant de performances extrêmement élevées.\medskip

  Selon Robinson \emph{el al.} \cite{robinson2013graph} \textit{``Un
    système de gestion de base de données orienté graphe (ou
    simplement, une base de données graphe) est un système de gestion
    de base de données en ligne avec les quatre
    opérations de base pour la persistance des données
    (\acrshort{crud}) exposant une structure de donnée
    graphe.''}\bigskip

  Angles et Gutierrez \cite{angles2008survey} conceptualise les
  bases de données graphe (\textit{``graph db-model''}) en trois
  éléments fondamentaux:\medskip
  \newpage
  \renewcommand{\descriptionlabel}[1]{\hspace{1cm}\textbullet~\textsf{#1}}
  \begin{description}
  \item [Modèle de données]: une base de donnée graphe permet la
    modélisation, le stockage et la manipulation des données complexes
    liées par des relations non-triviales ou variables. Elle stocke
    les données en graphe ou en une structure de données plus générique
    (\textit{hypergraphs} ou \textit{hypernodes}) capable de
    représenter avec élégance n'importe quel type de données d'une
    manière ultra accessible. Un graphe contient des \textit{nœuds} ,
    reliés par des \textit{relations}. Dans le modèle de graphe à
    propriétés, chaque nœud et relation peut être libellée et contenir
    plusieurs \textit{propriétés} les décrivant. Les relations entre
    les nœuds sont un élément clé d'une base de données graphe, elles
    permettent de trouvent les données reliées, tout comme les
    noeuds.\medskip

  \item [Manipulation de données]: Les manipulations de données sont
    exprimées par des transformations de graphe ou des
    \textit{traversées}, ces opérations utilisent les primitives
    fondamentales d'une structure de données graphe, comme les
    relations d'adjacence entre les nœuds, les chemins de parcours,
    les sous-graphes et  la notion de connectivité, etc. Ces opérations
    sont utilisées pour la création, l'interrogation et la mise à jour
    d'une base de données graphe. Une \textit{traversée} consiste à
    la manière dont les données sont interrogées dans un graphe,
    naviguant depuis les nœuds de départ vers les noeuds reliés en
    accordance avec un algorithme.\medskip

  \item [Les contraintes d'intégrité]: une contrainte d'intégrité
    permet de garantir la cohérence des données lors des mises à jour,
    ces contraintes nous permettent, par exemple, de typer des nœuds,
    définir des contraintes d'unicité sur certains champs, etc.
  \end{description}
\enddescription

  \subsection{Techniques de persistance des bases de données graphes}
  \label{sec:persistence}
  Généralement, nous pouvons distinguer trois techniques majeures de
  stockage adoptés par les systèmes de gestion des bases de données
  graphe: les bases de données graphe qui utilisent des
  \acrshort{SGBDR} comme un \emph{backend}, celles qui utilisent des
  systèmes \acrshort{nosql}, et enfin, nous avons les bases de données
  graphe natives qui fournissent ses propres implémentations de
  stockage.

    \subsubsection{Bases de données graphe au-dessus d'un stockage
      SQL}
    \label{sec:graphdb-over-sql}
    Une base de donnée graphe peut être stockée dans une base de
    données relationnelle. Les étiquettes et les attributs de nœuds et
    arcs peuvent être gérés séparément dans d'autres tables et renvoyés
    par des clés étrangères. l'utilisation des \acrshort{SGBDR} comme
    un moteur de stockage a quelques avantages : des systèmes
    d'indexation évolués, un support des transactions sophistiqué, et
    le langage de requêtes \emph{SQL} qui est un standard bien établi
    avec un cycle d'apprentissage rapide.\medskip

    \input{figs/periscope-gq.tex}

    Cette classe comprend \emph{GRIPP} \cite{trissl2007fast}, et
    Periscope/GQ \cite{tian2008periscope} qui implémente un système de
    gestion des bases de données graphe comme une application
    au-dessus d'un moteur de stockage relationnel \emph{PostgreSQL}.

    \newpage
    \subsubsection{Bases de données graphes au-dessus d'un stockage
      NoSQL}
    \label{sec:graphdb-over-nosql}
    Plusieurs systèmes de base de données graphe utilisent des
    systèmes \acrshort{nosql} comme un moteur de stockage interne
    offrant une meilleure scalabilité et un support fiable pour le
    partitionnement des données.\bigskip

    \textsf{HypergraphDB} \cite{hypergraphdb,
      iordanov2010hypergraphdb} est une base de donnée qui implémente
    le modèle de données \emph{``hypergraph''} où la notion d'un arc
    est étendue pour pouvoir connecter plus de deux nœuds, ce qui est
    particulièrement utile pour la modélisation des données telles que
    la représentation des connaissances, l'intelligence artificielle
    et la bio-informatique. \emph{HypergraphDB} utilise le modèle
    \textit{clé/valeur} de \emph{BerkeleyDB} \cite{berkeleydb} pour
    stocker toutes les informations relatives au graphe sous forme de
    pairs \textit{clé/valeur}, chaque objet du graphe (nœud ou arc)
    est identifié par une clé unique (appelée atome). Chaque atome est
    lié à un ensemble d'atomes par une relation de type \emph{0:N}
    (zéro ou plusieurs atomes), ces relations forment également la
    structure typologique ``hypergraph''.\bigskip

    \textsf{OrientDB} \cite{orientdb} est un système hybride de
    gestion de base de données graphes qui combine les fonctionnalités
    d'une base orientée document et une base orientée graphe avec
    une capacité de stockage des données structurées ou
    semi-structurées (\emph{schema-less}). il supporte aussi la
    répartition de charge à travers plusieurs machines et la
    réplication multi-maîtres tout en assurant les propriétés
    \acrshort{acid} des données. Pour les requêtes simples,
    \emph{OrientDB} s'appuie sur \emph{SQL} et utilise des langages de
    parcours des graphes comme \emph{Gremlin} afin d'éviter les
    jointures \emph{SQL} coûteuses pour les requêtes
    complexes.\bigskip

    \textsf{Titan} \cite{titan} est une base de données orientée
    graphe évolutive (\emph{scalable}) et transactionnelle. Elle est optimisée
    pour le stockage et l'interrogation des données graphes contenant
    des centaines de milliards de sommets et d'arcs à travers un
    \emph{cluster} multi-machine avec des schémas complexes du
    parcours et requêtage et une exécution en temps réel. \emph{Titan}
    utilise une multitude de systèmes \emph{NoSQL} comme un moteur de
    stockage (\emph{backend}), par exemple, \emph{Hbase},
    \emph{Cassandra}, \emph{BerkeleyDB}, cette diversité offre une
    flexibilité en terme des caractéristiques \emph{CAP} \ref{sec:cap}
    assurées par le système.

    \subsubsection{Les bases de données graphe natives}
    \label{sec:graphdb-native}
    Les bases de données graphe natives possèdent leur propre
    système de fichiers pour stocker les données au lieu de compter
    sur des moteurs de stockage tiers. Ces bases de données sont
    optimisées pour stocker et gérer les données \textbf{fortement}
    connectées où la performance et la disponibilité sont
    primordiales.\bigskip

    % TODO: enhance
    \textsf{Neo4j} \cite{neo4j} est un système de gestion de base de
    données graphe \textit{open-source} populaire implémenté en
    \textit{java} et supporté commercialement par \textit{Neo
      Technology}\footnote{\url{http://neo4j.com/}}. il a été
    conceptualisé et construit depuis le début afin d'être un système
    fiable, robuste, évolutif et très performant optimisé pour stocker
    et gérer des grandes masses de données structurées en graphe,
    \textit{Neo4j} convient aussi bien pour le déploiement en grande
    entreprise pour des projets plus légers en utilisant une
    partie du serveur. Parmi ses fonctionnalités:\medskip

    \begin{itemize}\renewcommand\labelitemi{--}
    \item Transactions \acrshort{acid} réelle avec une haute disponibilité.
    \item Supporte jusqu'à des milliards de noeuds et relations.
    \item Requêtes ultra-rapides grâce aux traversées.
    \end{itemize}
    \enddescription
    \medskip

    \textit{Neo4j} est fourni avec une \textit{API} basée sur des
    \textit{callbacks}, ce qui laisse la possibilité de spécifier
    les règles pour les traversées, Les autres options pour traverser
    ou interroger un graphe dans \textit{Neo4j} sont \textit{Gremlin}
    \ref{sec:gremlin} et \textit{Cypher} \ref{sec:cypher}.\bigskip

    \textsf{AllegroGraph} \cite{allegrograph} est un système performant
    de persistance des données graphes avec un stockage natif,
    implémenté initialement comme un système de persistance et
    gestion de documents \acrshort{rdf} doté d'une interface
    \acrshort{rest} \cite{fielding2000architectural}. Similaire à
    \emph{Neo4j}, \emph{Allegrograph} garantit la satisfaction des
    propriétés \acrshort{acid},  d'atomicité, cohérence, isolation et
    durabilité. \emph{AllegroGraph} supporte \acrshort{sparql}, RDFS++
    et \textit{Prolog} pour des applications clientes
    multiples.\bigskip

    \textsf{Sparsity} \cite{sparksee} (auparavant connu sous le nom
    \emph{DEX}) est un système natif de stockage de données graphe
    persistantes et temporaires. \emph{Sparsity} focalise sur la
    gestion et l'interrogation à haute performance des bases de 
    données graphes de grande taille. En encodant les matrices d'adjacences en
    \emph{bitmap}, le système a une gestion d'espace disque compacte et
    efficace et assure une très bonne performance.\medskip
    \newpage

    \textsf{InfiniteGraph} \cite{infinitegraph} est un système
    distribué de gestion des bases de données graphe qui supporte des
    grandes masses évolutives de données (\emph{large-scale}) avec une
    capacité d'analyse efficace des graphes et une bonne prise en
    charge des algorithmes distribués du
    parcours. \emph{InfiniteGraph} fournit une variété
    d'implémentations pour des différents systèmes (serveurs
    d'application, les plateformes de \emph{cloud computing} et les
    systèmes embarqués).

  \subsection{Techniques de stockage et d'indexation}
  \label{sec:graph-internals}
  Il ya plusieurs façons pour représenter un graphe dans une mémoire
  principale d'un système de stockage. On dit qu'un système de gestion
  des bases de données graphe dispose des capacités \emph{natives} de
  parcours des graphes s'il possède une propriété fondamentale
  appelée \emph{index-free adjacency}.

  \input{figs/gpi.tex}
  \newpage

    \subsubsection{Index-free adjacency}
    \label{sec:index-free}
    Dans un système de stockage de base de données graphe qui utilise
    la technique \emph{index-free adjacency}, chaque noeud maintient
    des références directes à ses noeuds adjacents agissant comme un
    micro index d'autres noeuds voisins. Cette méthode améliore la
    fiabilité du système de stockage d'une façon considérable
    comparant à celles qui utilisent un index global. En effet, le temps
    de réponse des requêtes est indépendant de la taille globale du
    graphe. Pour traverser un graphe d'une taille donnée $n$ à $m$
    pas, le coût est d'ordre $\mathcal{O}(m\log{}n)$ dans un système
    avec un index global, et d'ordre $\mathcal{O}(m)$ pour un système
    natif qui utilise un \emph{index-free adjacency}.

    \input{figs/vertex-centric-indices.tex}

    \subsubsection{Vertex Centric Indices}
    \label{sec:vertex}

    L'objectif de cette technique est de trier et indexer tous les arcs
    incidents d'un noeud (et donc les noeud adjacents) selon les
    propriétés de ces arcs. Ces indexes éliminent la nécessité de
    balayage linéaire des arcs incidents lors d'une requête de parcours
    ($\mathcal{O}(n)$) en faveur d'un parcours plus rapide
    ($\mathcal{O}(\log{}n)$). Cette méthode est utilisée par (\emph{Titan}
    \cite{vertexci}) surtout pour confronter le problème de
    \emph{Supernode} dans les graphes fortement connectés, un
    \emph{Supernode} est un noeud avec un très grand nombre d'arcs
    incidents.

  % \newpage
  % \subsection{Comparaison}
  % \label{sec:graphdb-comp}
  % \input{tables/graphdb-comp.tex}
  % \newpage

  % {\color{red}
  %   to be continued
  % }
  % \newpage

\section{Langages des requêtes}
\label{sec:query-languages}
Les langages des requêtes ont toujours été la clé du succès des
\acrshort{SGBD}. La prévalence des \emph{\acrshort{SGBDR}} dans les
dernières décennies est étroitement couplée avec le succès du
\emph{SQL}. Des divers langages ont été définis pour exprimer des
requêtes vis-à-vis plusieurs dépôts de données, par exemple,
\emph{XQuery} \cite{boag2002xquery} et \emph{XPath}
\cite{clark1999xml} pour les bases de données \emph{\acrshort{xml}},
\emph{QQL} \cite{alashqur1989oql} pour les bases de données orientées
objet et \acrshort{sparql} \cite{prud2008sparql} pour les
triplestores (les bases de données \emph{RDF}). Cette section présente
trois langages des requêtes largement supportés par les systèmes de
gestion des bases de données graphe.

% \input{figs/query-graph-example.tex}

% Le graphe présenté par la figure \ref{fig:query-graph-example} sera
% utilisé pour illustrer la syntaxe et la sémantique de ces trois
% langages. Le graphe représente le résultat d'un processus de
% \emph{matching} sémantique entre cinq services Web qui forment un plan de
% composition.

  \subsection{SPARQL}
  \label{sec:sparql}
  \acrshort{sparql} \cite{prud2008sparql} est un langage populaire de
  requête pour les données \acrshort{rdf}, il est reconnu comme l'une
  des technologies clés du Web sémantique. Le standard
  \acrshort{sparql} est largement utilisé pour exprimer des
  interrogations à travers diverses sources de données graphe comme
  des triplestores \acrshort{rdf}. Il est capable de rechercher des
  motifs de graphe (\emph{graph patterns}) ainsi que leurs
  conjonctions et leurs disjonctions. Les résultats des interrogations
  \acrshort{sparql} peuvent être des ensembles de résultats ou des
  graphes \acrshort{rdf} qui peuvent être retournés via
  \acrshort{http} dans une variété de formats tels que \acrshort{xml},
  HTML ou \acrshort{json}\medskip

  \acrshort{sparql} propose quatre types de requête:\medskip

  \begin{itemize}\renewcommand\labelitemi{--}
  \item les requêtes \texttt{SELECT} permettent d'extraire des
    informations d'une base de données ou d'une source de données
    \acrshort{rdf} interrogée.

  \item les requêtes \texttt{CONSTRUCT} permettent de créer de
    nouveaux triplets à partir du résultat d'une requête.

  \item les requêtes \texttt{DESCRIBE} permettent d'obtenir la
    description d'une ressource donnée sous forme d'un
    sous-ensemble de graphes interrogés.

  \item \texttt{ASK} indique si la requête retourne un résultat non
    vide
  \end{itemize}
  \enddescription
  \medskip

  La plupart des bases de données graphes ont un support de
  \textsc{SPARQL} soit nativement telle que \emph{Allegrograph} soit
  via des extensions comme \emph{Neo4j}.

  \subsection{Gremlin}
  \label{sec:gremlin}
  \emph{Gremlin} \cite{gremlin-wiki} est un langage de domaine
  spécifique (\acrshort{DSL}) de bas niveau pour le parcours des
  graphes attribués. Il trouve ses applications dans les domaines de
  la recherche, l'analyse et la manipulation des bases de données
  orientées graphe qui implémentent le modèle \emph{Blueprints}
  \cite{blueprints} de données. \emph{Gremlin} \cite{gremlin-wiki} est
  un projet open source développé et maintenu par
  \emph{TinkerPop}.\medskip

  La syntaxe \emph{Gremlin} est basée sur \emph{XPath} de manière à
  être capable d'exprimer des descriptions de parcours même profonds
  avec des expressions simples et compactes.\medskip

  La distribution \emph{Gremlin} (maintenu par \emph{TinkerPop}) est
  supportée par la plupart des bases de données graphe via des
  langages \emph{JVM} comme \emph{Java}, \emph{Groovy} et
  \emph{Scala}, parmi ces \acrshort{SGBD} graphe on trouve
  \emph{Neo4j}, \emph{Titan} et \emph{OrientDB}.

  \subsection{Cypher}
  \label{sec:cypher}
  \emph{Cypher} \cite{cypher-docs} est un langage de requête
  déclaratif pour interagir avec les bases des données graphes
  \emph{Neo4j}, développé et maintenu par \emph{Neo Technology}. Il
  permet d'effectuer des requêtes et mises à jour efficaces du graphe 
  sans avoir à écrire des traversées (parcours) d'une manière
  procédurale.\medskip

  En étant un langage déclaratif, \emph{Cypher} se concentre sur la
  clarté d'exprimer \textit{quoi retrouver dans un graphe et non
  comment le faire}. Ceci est en contraste aux langages impératifs
  comme Java et aux langages script comme \emph{Gremlin}
  \ref{sec:gremlin}, ce qui rend le fait d'optimisation des requêtes un
  détail d'implémentation non exposé aux utilisateurs.\medskip

  \emph{Cypher} est inspiré de plusieurs approches et construit sur
  des pratiques établies pour l'interrogation expressive des bases de
  données. La plupart des mots clés comme \verb|WHERE| et
  \verb|ORDER BY| et La concordance des patterns sont hérités
  directement des langages déclaratifs comme \emph{SQL} et
  \emph{SPARQL} \ref{sec:sparql} avec quelque propriétés inspirées des
  langages fonctionnels comme \emph{Haskell} et \emph{ML}.\medskip

  Le langage \emph{Cypher} comporte un nombre de clauses distinctes,
  des clauses pour l'interrogation du graphe comme:
  \begin{itemize}
  \item [\texttt{MATCH}]: Utilisé pour décrire Le pattern du graphe
    à correspondre, principalement sur la base des relations entre les
    nœuds du graphe.
  \item [\texttt{WHERE}]: Sert à un critère de filtrage.
  \item [\texttt{RETURN}]: Spécifie ce qu'il faut retourner comme
    résultat finale à la requête.
  \end{itemize}
  \medskip

  \emph{Cypher} contient en outre des clauses pour l'écriture, la mise
  à jour et la suppression des données, par exemple:

  \begin{itemize}
  \item [\texttt{CREATE}]: Crée des nœuds ou des relations.
  \item [\texttt{SET}]: Affecte des valeurs aux propriétés..
  \item [\texttt{DELETE}]: supprime des noeuds, relations ou propriétés.
  \end{itemize}

\section*{Conclusion}
\addcontentsline{toc}{section}{Conclusion} \markboth{CONCLUSION}{}

Dans ce chapitre nous avons fait un tour d'horizon des systèmes de
gestion des bases de données graphe conçues pour modéliser des
réseaux de données fortement connectées, tout en mettant l'accent sur
les techniques natives de stockage et d'indexation adoptées par des
solutions très répondus dans l'industrie comme \textit{Neo4j} et
\textit{AllegroGraph}.

% TODO
% Nous avons pu constater que ...

%%% Local Variables:
%%% mode: latex
%%% TeX-master: "../main"
%%% End:
