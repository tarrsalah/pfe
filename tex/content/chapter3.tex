\chapter{Les approches de composition dynamique des services Web sémantiques baseés sur le modèle graphe}
%% Introduction to the chapter
\newpage
\section{Contexte}
%% positionner l'approche par graphe dans les classification déjà
%% abordées
  \subsection{Définitions et terminologies}
  \label{sec:defin-et-term}
  Le matching, l'alignement, le mapping...ect. Sont tous des termes
  utilisées pour référencer le concept de recherche de
  similarité. Voici quelques définitions de ces termes utilisés.

  \begin{mydef}[Matching]
    c'est le processus de découverte des liaisons et des
    correspondances entre les entités de différentes représentations à
    travers un algorithme de matching.
  \end{mydef}

  \subsection{Matching des services Web}
  \label{sec:match-des-serv}

  \subsection{Mésure de similariré}
  \label{sec:mesure-de-similarire}

  \subsection{Graphe de dépandence}
  \label{sec:graph-de-depandence}

\section{Travaux relatives}
\label{sec:travaux-relatives}

% La découverte d’un service composite à partir d’un graphe de
% dépendance consiste à trouver le meilleur chemin dans le graphe qui
% génère les sorties exprimées dans la requête. La recherche peut se
% baser sur l’optimisation d’une fonction d’utilité qui tient compte du
% degré de matching entre les services et aussi de la qualité de service
% qui peut être fournie par les services concrets. A cet effet, les
% méthodes de recherche du meilleur chemin de la théorie des graphes
% peuvent être exploitées.  Les méthodes de composition utilisant le
% modèle de graphe peuvent être classées en deux catégories selon que le
% graphe de dépendance soit construit à priori (en phase de publication
% des services) ou pendant le traitement de la requête de composition :

  \subsection{Génération Online du graphe de dépendance}
  \label{sec:generation-online-du}

  \subsection{Génération Offline du graphe de dépendance}
  \label{sec:gener-offl-du}

  \subsection{Autres travaux basés sur le graphe matching}
  \label{sec:autres-travaux}

\section{Vers les bases de données graphe}
\label{sec:vers-les-bases}

\section{Conclusion}
\label{sec:conclusion-2}


%%% Local Variables:
%%% mode: latex
%%% TeX-master: "../main"
%%% End: