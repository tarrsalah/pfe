%!TEX root = ../main.tex
% < 20 pages

% \chapter{Concepts de base du paradigme service Web}
% Une introduction implicite du chapitre %TODO
    % Dans ce chapitre on va faire une vue générale sur les technologies des services web sémantiques, Pour bien 
    % introduire le chapitre prochain sur la composition des services web.

    Ce chapitre établit une étude du fondement théorique de notre travail à savoir les concepts de base du paradigme
    service Web.  Nous commençons d'abord par présenter un tour d'horizon définissant l'infrastructure et
    l'architecture de référence de ce paradigme ainsi que quelque définitions proposées par la littérature. Ensuite
    nous nous intéressons à montrer les limitation de l'approche syntaxique de la description des services web et l'apport
    de l'enrichissement sémantique de cette dernière aux processus de la découverte et la composition des services Web.\\
    % TODO: complete the introduction

    \newpage
    
\section{Les services web : Notions de base et technologies associées} 
    Les services Web constituent une approche pour mettre en œuvre le paradigme de service, et peut être vue comme
    une instance de l'architecture orienté service.

    Dans cette section va parler aussi d'un socle technologique très sollicité, On va aussi Détailler l'architecture de base 
    d'un service web, ensuite nous introduisons l'architecture étendus.

    \subsection{Définitions et caractéristiques}
	Les services Web sont la technologie la plus connue et la plus populaire dans le monde industriel et
	académique pour la mise en place d’architectures à services. 
	% ARE PROGRAMMABLE COMPONENTS which use the World-
	% Wide-Web as a medium for describing the functionality of real world ser-
	% vices in a computer manipulable way (Kuropka and Nern, 2006)
	Les Web services ont été proposés initialement par IBM et Microsoft, puis en partie standardisés par le 
	consortium du World Wide Web (\emph{W3C}) et définis \cite{w3c_ws_arch:2014:Misc} par:

	\emph{``Un service web est un système conçu pour permettre d'interopérabilité des applications à travers un réseau. 
	    Il est caractérisé par un format de description interprétable/compréhensible automatiquement par la machine,
	    D'autres systèmes peuvent interagir avec le Service Web selon la manière prescrite dans sa description et en
	    utilisant des messages SOAP, généralement transmis vie le protocole HTTP et sérialisés en XML et en d'autres 
	    standards du Web ''}.\\

	From the definition, Web services are the following attributes:
	% REF [ENTERPRISE SERVICES] entrepriseServices.pdf

	% TODO to translate
	\begin{description} % Les caractéristiques des services web
	    \item[Web-based protocols] : 
		Web services based on SOAP-over-HTTP are designed to work over the public Internet. 
		The use of HTTP for transport means these protocols can traverse firewalls, and can work 
		in a heterogenous environment.

	    \item[Interoperability] : 
		SOAP defines a common standard that allows differing systems to interoperate. For example, the tooling 
		allows Visual Basic clients to access Java server components and vice versa.

	    \item[XML-based] : 
		The Extensible Markup Language is a standard framework for creating machine-read- able documents. 
		Managed by the World Wide Web Consortium (www.w3.org), XML is an open Web standard for creating 
		interoperable standard documents.
	\end{description}
	\newpage
	M. P. Papazoglou \cite{papazoglou2003service} apporte une autre définition de services web:\\

	\emph{``Les services Web sont des éléments auto-descriptifs et indépendants des plateformes qui permettent 
	    la composition faible coût d’applications distribuées. Les services Web effectuent des fonctions allant 
	    de simples requêtes des processus métiers complexes. Les services Web permettent aux organisations d’exposer 
	    leurs programmes résultats sur Internet (ou sur un intranet) en utilisant des langages (basés sur XML)
	    et des protocoles standardisés et de les mettre en œuvre via une interface auto-descriptive basée sur 
	    des formats standardisés et ouverts''}\\
	% TODO make a comment on this def and introduce the web services composition idea !

	L'approche de service Web vise essentiellement quatre objectifs fondamentaux expliquant son grand succès:\\
	\begin{description} % Les Objectives des services web
	    \item[L'interopérabilité] : L'interopérabilité permet des applications écrites dans des langages 
		de programmation différents et s’exécutant sur des plateformes différentes de communiquer entre elles. 
		En manipulant différents standards que ce soit XML ou les protocoles d’Internet, les services Web garantissent
		un haut niveau d’interopérabilité des applications et ceci indépendamment des plateformes sur lesquelles 
		elles sont déployées et des langages de programmation dans lesquels elles sont écrites. Ainsi,
		en s’appuyant sur un format d’échange de messages standard et sur l'ubiquité de l'infrastructure d’Internet,
		l'interopérabilité est donc une caractéristique intrinsèque aux services Web.

	    \item[Le couplage faible] : Le couplage est une métrique indiquant le niveau d’interaction entre deux ou plusieurs
		composants logiciels. Deux composants sont dits couplés s’ils échangent de l'information. Nous parlons 
		de couplage fort si les composants échangent beaucoup d’information et de couplage faible dans le cas contraire. 
		Vu que la communication avec les services Web est réalisée via des messages décrits par le standard XML 
		caractérisé par sa généricité et son haut niveau d’abstraction, les services Web permettent la coopération 
		d’applications tout en garantissant un faible taux de couplage.  Par conséquent, il est possible de modifier
		un service sans briser sa compatibilité avec les autres services composant l'application.

	    \item[La réutilisation] : L'avantage de la réutilisation est qu'elle permet de réduire les coût de 
		développement en réutilisant des composants déjà existants. Dans le cas de l'approche service Web, 
		l'objectif de la séparation des opérations en services autonomes est en effet pour promouvoir leur 
		réutilisation. Ainsi, lorsqu'un client définit ses exigences, il est généralement possible de réutiliser
		des services déjà existants pour satisfaire une partie des exigences. Ceci facilite la maintenance de
		l'application et permet un gain de temps considérable.

	    \item[La découverte et la composition automatiques] : 
		La découverte et la composition sont des étapes importantes qui permettent la réutilisation des services. 
		En effet, il faudra être en mesure de trouver et de composer un service afin de pouvoir en faire usage. 
		En exploitant les technologies offertes par Internet et en utilisant un ensemble de standards pour la 
		publication, la recherche et la composition, l'approche services Web tend ` diminuer autant que possible 
		l'intervention humaine en vue de permettre une découverte et une composition automatiques des services
		les plus complexes. En effet, pour réaliser son application, un développeur peut simplement interroger 
		un moteur de recherche de services afin de trouver le service adéquat et à l'aide de langages de 
		coordination appropriés il peut l'intégrer avec le reste des services de son application.
	\end{description}

	Un service Web est une application réseau capable d'interagir par le moyen des standards et des protocoles 
	via des interfaces bien spécifiés, dans lequel est décris utilisant un langage de description fonctionnel
	standardisé \cite{curbera2001web}.\\

	% Brièvement, un service Web est une entité logiciel modulaire, auto-descriptive et autonome accessible 
	% via Internet.

    \subsection{L'évolution des styles des services web}
    % Les architecture communes des services web
	% SOAP vs REST
	% pourquoi SOAP ?
	% les limites de la'approche SOAP 
	``the next section provides a short history of web services, with emphasis on the kinds of software challenges
	that web services are meant to address.''

    \subsection{L'architecture de référence et technlogies associées}
    %TODO the main arch goes here !
	Les services Web sont construits autour de standards qui sont SOAP, WSDL et UDDI assurant respectivement leur
	communication, leur description et leur découverte .

	\renewcommand{\descriptionlabel}[1]{\hspace{1cm}\textbf{#1}}
	\begin{description}
	    \item[SOAP] : 
		\textit{Simple Object Access Protocol} est un protocole simple et léger basé sur \textsc{XML} spécifié 
		par le \textsc{W3C} \cite{soap_1.2_primer:2014:Misc}. Il assure la communication entre clients et services 
		par échange de messages au travers du Web.  Il utilise principalement le protocole \textsc{HTTP(S)} 
		(\textit{HyperText Tranfer Protocol}) pour le transport de messages.\\

		Un message SOAP est composé de:
		\renewcommand{\descriptionlabel}[1]{\hspace{1cm}\textsf{#1}}
		\begin{description} 
		    \item[En-tête]: 
		    \item[Corps]: 
		\end{description}
	    \item[WSDL] : Le WSDL composé de plusieurs 
	    \item[UDDI] : UDDI est un annuire des services
	\end{description}

     %conclure cette subsection par la mention du problème de la découverte automatique des services Web et l'insuffisance
     \newpage

\section{Description des services web} 
    % introduction %TODO
    La description d’un service consiste à définir une interface exposant les opérations accomplies par le service et 
    lier chaque opération à sa réalisation. Dans cette section nous présentons les modèles de description des services
    	\subsection{Description syntaxique de services}
	% WSDL 1.1 , WSDL 2.0
        \subsection{Ajout de la sémantique}
	    % la description syntaxique est insuffisante.
	    A Semantic Web service is defined as an extension of Web service description through the Semantic Web annotations,
	    created in order to facilitate the automation of service interactions . Therefore, from 
	    he perspective of the functionality offered, Semantic Web services are still Web services. The only difference lays
	    in their description and the consequent benefits that follow, namely the reduction of human involvement in 
	    he performed interactions.\\

	    - insuffisance de description syntaxique des services web :(WSDL)
	     % What is semantic web? \\
	     RDF? \cite{lassila1999resource}\\
	    ``Ontologie is a representation of a shared conceptualisation of of a particular domain'' 
	     % \cite{decker2000semantic}\\

	\subsubsection{Annotations sémantiques};
	  % WSDL-S
	  % SAWSDL
	\subsubsection{Ontologies de services} 
	  % OWL-S
	  \cite{mcguinness2004owl} , \cite{mcilraith2003bringing}
	  % WSMO

\section{Découverte des services web}
   % parler de l' UDDI,le matching sémantique !!


\section{Conclusion}
    % faire un petit récapitulatif sur les technologies des services web 
    % rappeler de notre problème principale : composition des services web
