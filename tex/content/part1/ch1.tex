%!TEX root = ../main.tex

\chapter{Les services web :Vue d'ensemble}
Introduction du chapitre %TODO

    \section{L'orientation service}
    	L'approche à services est une approche relativement récente qui présente un certain nombre d'avantages pour 
	la réalisation des applications. Nous allons, dans un premier temps, définir l'élément-clé de l'approche 
	à services, c'est-à-dire le service. Ensuite, nous présenterons les différents acteurs et leurs interactions 
	pour cette approche. Dans une troisième partie, nous détaillerons l'architecture à services. Nous expliquerons 
	aussi les besoins qui en découlent. Finalement, nous mettrons en évidence les éléments spécifiques qui nous 
	permettront par la suite de caractériser les différentes technologies qui mettent en œuvre cette approche.\\

	Dans l'introduction de cette section on va définir,  les termes suivants: \emph{``Service, SOC et SOA ''}\\
	    * C'est quoi un service?\\

	\subsubsection{Service}
	    Plusieurs définition sont proposées par la littérature:\\

	    * M.P Papazoglou proposé la définition suivante :\cite{Papazoglou:what_is_a_service?} ``Software 
	    services (or simply service are self-contained, platfor-agnostic computational elements that support 
	    rapid, low-cost and easy composition of loosely coupled distributed software applications''\\	;
	    
	    Une service est vue comme une entité logicielle qui peut être utilisé grâce à se description, 
	    qui peut être invoqué par le consommateur. \\

	    Selon \cite{Papazoglou:2007:SOA:1265289.1265298} ``Selon M. P. Papazoglou : ``Les services Web sont 
	    des éléments auto-descriptifs  et indépendants des plateformes qui permettent la composition à faible 
	    coût d'applications  distribuées. Les services Web effectuent des fonctions allant  de simples requêtes à  
	    des processus métiers complexes. Les services Web permettent aux organisations d'exposer  leurs programmes 
	    résultats sur Internet (ou sur un intranef) en utilisant des langages  (basés sur XML) et des protocoles 
	    standardisés et de les mettre en œuvre via une interface auto-descriptive basée sur des formats 
	    standardisés et ouverts''.''\\

	    Selon \cite{Papazoglou:2007:SOA:1265289.1265298} ``The emergence of Web services developments and 
	    standards in support of automated business integration has driven major technological advances in the 
	    integration software space, most notably, the service-oriented architecture (SOA)''.\\

	    Selon \cite{Fremantle:2002:ES:570907.570935} ``he purpose of this architecture is to address the 
	    requirements of loosely coupled, standards-based, and protocol-independent distributed computing, 
	    mapping enterprise information systems (EIS) appropriately to the overall business process flow.''\\
	    
       \subsubsection{helllo}
	  hello there.

    \section{Les services web: définitions et terminologies}
	\cite{srivastava2003web}

    \section{L'évolution des styles des services web}


    \section{L'étude du trio SOAP/WSDL/UDDI}
	Dans cette section va parler d'un socle technologique très sollicité.


    \section{Conclusion}
	Dans la conclusion on va parler sur les points suivants:\\
    \begin{itemize}	
	\item {rappeler de notre problème principale : composition des services web }
	\item {insuffisance de description syntaxique des services web :(WSDL)}
    \end{itemize}


    
